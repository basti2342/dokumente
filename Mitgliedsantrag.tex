\documentclass[a4paper,11pt]{scrartcl}
\usepackage[portrait,top=1.5cm,left=2.5cm,bottom=2cm,right=2.5cm]{geometry}
\usepackage[utf8]{inputenc}
\usepackage[T1]{fontenc}
\usepackage{textcomp} % für €
\usepackage{tabularx}
\usepackage{wasysym} % für \Square
\usepackage{multirow}
\usepackage{wrapfig}
\usepackage{graphicx}
\usepackage{url}

\urlstyle{sf}

\pagestyle{empty}
\setlength{\parindent}{0cm}
\setlength{\parskip}{0pt}
\setkomafont{minisec}{\normalfont\sffamily\bfseries}
\setkomafont{section}{\normalfont\sffamily\bfseries\Large}
\setkomafont{title}{\normalfont\sffamily\bfseries\huge}
\newcommand{\signskip}{\rule{0pt}{24pt}}
\newcommand{\smallsignskip}{\rule{0pt}{16pt}}
\newcommand{\hinweis}[1]{\emph{#1}}

\title{Mitgliedsantrag}

\begin{document}

% Header mit Logo und Anschrift
\newsavebox{\headerboxaddress}
\sbox{\headerboxaddress}{
  \parbox{.4\textwidth}{
    \small
    Vorstand des Freies~Labor~e.\,V. \\
    vorstand@freieslabor.org
  }
}
\newsavebox{\headerboxlogo}
\begin{center}
\begin{tabular}{@{}p{0.4\textwidth}@{\phantom{m}}p{0.4\textwidth}}
  \multicolumn{1}{r}{
    \usebox{\headerboxlogo}
  }
  &
  \usebox{\headerboxaddress}
\end{tabular}
\end{center}

% angepasster Titel... \maketitle verschwendet Platz
\begin{center}
  \vspace{\baselineskip}
  \Large \titlefont \makeatletter \@title \makeatother
  \vspace{0.5\baselineskip}
\end{center}

\hinweis{Mitgliedsanträge können auch formlos per E-Mail gestellt
werden, unter Angabe der hier aufgeführten Daten. Wir nehmen auch gerne
PGP-signierte oder -verschlüsselte Mails entgegen, unsere PGP-Schlüssel sind
unter {\upshape\url{???}} zu finden.}

\hinweis{Sobald der Vorstand über deinen Antrag entschieden hat, bekommst du per
E-Mail Nachricht über deine Aufnahme.}

% Persönliche Daten
\section*{Persönliche Daten}
\hinweis{Alle Änderungen dieser Daten müssen dem Vorstand gemeldet werden. Falls
der Vorstand dich in wich\-ti\-gen Fällen nicht erreichen kann, hat dies unter
Umständen deinen Aus\-schluss als Mitglied zur Folge.}

\minisec{Pflichtangaben}
(Bürgerlicher) Name: \hrulefill \smallsignskip \\
E-Mail-Adresse: \hrulefill \signskip

% Beitragsmodell
\section*{Beitragsmodell}
\minisec{Ordentliche Mitgliedschaft}
\begin{description}
  \item[\Square] Ich möchte den vollen Beitrag von \textbf{20€ pro Monat} zahlen.
  \item[\Square] Ich möchte den ermäßigten Beitrag von \textbf{10€ pro Monat}
    nach §0,~Abs.~2 der Beitragsordnung zahlen. (Ein entsprechender Nachweis
    muss dem Vorstand auf Verlangen zugänglich gemacht werden.)
  \item[\Square] Ich beantrage aus finanziellen Gründen nach §1,~Abs.~3 der
    Beitragsordnung eine individuelle Ermäßigung oder Befreiung. Diese gilt für
    ein Jahr und kann auf Antrag erneuert werden. \\
    \Square~Befreiung \hfill
    \Square~Ermäßigung auf: \hrulefill\signskip~€\hfill\phantom{a}
\end{description}

\minisec{Fördermitgliedschaft}
\begin{description}
  \item[\Square] Ich möchte Fördermitglied werden und einen frei wählbaren
    Beitrag von mindestens 30€ pro Jahr zahlen. \\
    Jährlicher Mitgliedsbeitrag: \hrulefill\signskip~€\hfill\phantom{a}
\end{description}
Datum, Unterschrift Mitglied: \hrulefill\smallsignskip

% Gesetzlicher Vertreter
\section*{Gesetzlicher Vertreter}
\hinweis{Dieser Abschnitt ist nur für minderjährige Antragsteller nötig.}

Name: \hrulefill \signskip \\
Datum, Unterschrift Vertreter: \hrulefill \signskip

\end{document}
% vim: set tw=80 et sw=2 ts=2:
