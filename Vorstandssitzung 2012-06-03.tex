\documentclass[a4paper,12pt]{scrartcl}
\usepackage[utf8]{inputenc}
\usepackage[T1]{fontenc}
\usepackage[ngerman]{babel}
\usepackage{libertine} % kann man notfalls auch ignorieren, wenns nicht da ist
\usepackage{textcomp} % für €
\usepackage[transcript]{stratum0doc}
\usepackage[colorlinks=false]{hyperref}

\addtolength{\textwidth}{-10pt}
\addtolength{\marginparwidth}{10pt}

\title{Vorstandssitzung des Stratum~0~e.~V.}
\date{3.~Juni~2012}

\begin{document}
\maketitle

\begin{description}
 \item[Anwesend:] Valodim (Vorsitzender) \\
    m00lean (stv.~Vorsitzender) \\
    rohieb (Beisitzer) \\
    larsan (Beisitzer) \\
		Ortwin (Beisitzer) \\
    sowie 5 Gäste ohne Stimmrecht
	\item[Abwesend:] Neo Bechstein (Schatzmeister)
  \item[Sitzung eröffnet] um 17:16
  \item[Protokoll und Versammlungsleitung:] rohieb (Wahl durch Handzeichen)
\end{description}

%%%%%%%%%%%%
%% TOP  0 %%
%%%%%%%%%%%%
\section{Streaming/Videoaufzeichnung der Sitzung}
\withdrawn
rohieb hat im Voraus den Antrag gestellt, die Sitzung aufzuzeichnen und wenn
möglich live über das Internet zu streamen, um mehr Transparenz zu erreichen. Da
aber keine Aufnahmegeräte vorhanden sind, zieht er seinen Antrag zurück.

%%%%%%%%%%%%
%% TOP  1 %%
%%%%%%%%%%%%
\section{Umlaufbeschlüsse}
\subsection{Mitgliedsanträge}
\vote{Mitgliedsanträge}{5}{0}{0}
Folgende Mitgliedsanträge sind seit der letzten Vorstandssitzung eingegangen,
ihnen wird ohne Gegenstimmen und Enthaltungen zugestimmt:
\begin{itemize}
  \item 2012-04-06: Dennis Velone
  \item 2012-04-08: Jan Kretschmer
  \item 2012-04-12: Jonas Martin
  \item 2012-04-20: Lena Maria Schimmel
  \item 2012-04-24: Miguel Godinho
  \item 2012-05-04: Simon Kartheuser
  \item 2012-05-28: Georg von Zengen
\end{itemize}

Es wird zu Protokoll gegeben, dass Roman Krüger am 6. April 2012 seinen
Austritt zum Ende jenes Monats erklärt hatte.

\subsection{Erstattung 38{,}91€ Raspberry Pi}
\vote{Erstattung 38{,}91€ Raspberry Pi}{5}{0}{0}
rohieb hat 38{,}91€ für einen Raspberry Pi ausgegeben und würde diesen gegen
Erstattung des Betrags dem Verein übereignen. Der Antrag wird ohne Gegenstimmen
und Enthaltungen angenommen. Der Raspberry Pi soll für die Mitglieder zur
freien Verfügung stehen.

%%%%%%%%%%%%
%% TOP  2 %%
%%%%%%%%%%%%
\section{Finanzen}
\subsection{Finanzbericht}
\consensus{Konto: 2{.}200€ (geschätzt), Barkasse: 500-600€ (geschätzt)}
Da Neo Bechstein nicht anwesend ist, gibt Valodim den aktuellen Stand der
Finanzen aus dem Gedächtnis wieder. Aktuell befinden sich etwa 2{.}200€ auf dem
Konto (unzuverlässige Angabe, geschätzt), dies entspricht auch ungefähr dem
finanziellen Puffer, der bei der letzten Vorstandssitzung am 5. Februar 2012
beschlossen wurde zurückzulegen. In der Barkasse befinden sich etwa 500-600€.

\subsection{Antrag: Projektgeldverteilung im Plenum besprechen}
\vote{Verteilung von Projektgeldern im Plenum}{5}0{0}
rohieb stellt den Antrag, die Verteilung von Projektgeldern im Plenum zu
besprechen, um auch anderen Mitgliedern die Gelegenheit zu geben, sich zu der
Verteilung ihrer Mitgliedsbeiträge zu äußern. Die Beschlüsse des Plenums müssen
wie bisher durch den Vorstand bestätigt werden. Der Vorschlag stößt auf breite
Zustimmung und wird ohne Gegenstimmen und Enthaltungen angenommen.

\subsection{Stromverbrauch}
\consensus{Zwischenzähler einbauen}
Daniel Willmann hatte im Vorfeld darauf hingewiesen, den Stromverbrauch zu
beobachten und ggf. die Schätzung für die Zukunft anzupassen, da sich inzwischen
der Bestand an Elektrogeräten, die im Hackerspace betrieben werden, gegenüber
den ersten Monaten erhöht hat. Da keiner der anwesenden Personen weiß, wo sich
der Stromzähler für unsere Räume befindet, aber anscheinend feststeht, dass eine
Ablesung nur nach Kontakt zum Hausmeister möglich ist, stößt der Vorschlag,
einen Zwischenzähler im Sicherungskasten einzubauen, auf breite Zustimmung.

\subsection{Antrag: Erstattung 50€ für Projekt "`Hüfüwagen"'}
\vote{Erstattung 50€ für Hüfü\-wagen}{5}{0}{0}
Neo Bechstein hatte einen Antrag auf Erstattung von 50€ für das Projekt
"`Hüfüwagen"' gestellt. Darin enthalten sind sein Bollerwagen, eine
Autobatterie inkl. Ladegerät und ein Audio-Verstärker. Der Antrag wird ohne
Gegenstimmen oder Enthaltungen angenommen.

\subsection{Antrag: Geld für Moodlights}
\postponed
Von m00lean stammt der Antrag, Geld für Moodlights bereitzustellen. Da aber noch
keine Schätzung der Kosten existiert, wird dieser Punkt vertagt. Generell
spricht Valodim sich dafür aus, zuerst Angebote einzuholen, und dann Anträge auf
Projektgelder zu stellen.

\subsection{Aufkleber}
\vote{Antrag 48{,}27€ für Aufkleber}{5}{0}{0}
rohieb findet es sinnvoll, Aufkleber mit dem Vereinslogo und der URL zur
Webseite drucken zu lassen. Über das endgültige Design soll im nächsten Plenum
abgestimmt werden. Er schlägt vor, die Aufkleber by der Firma flyeralarm zu
bestellen und beantragt 48{,}27€ für die Anfertigung von 1000 Stück. Der
Antrag wird ohne Gegenstimmen oder Enthaltungen angenommen.

\subsection{3D-Drucker}
\consensus{Finanzierung 3D-Drucker über Crowdfunding, not\-falls Zu\-schüs\-se
aus Vereins\-mitteln}
Zur Zeit befindet sich noch der von Daniel Willmann bereitgestellte 3D-Drucker
im Hackerspace und kann von Mitgliedern nach Unterrichtung über die
Funktionsweise benutzt werden. Daniel wird jedoch in absehbarer Zukunft den
3D-Drucker wieder mitnehmen, sodass die Anschaffung eines vereinseigenen
Druckers sinnvoll erscheint.

Es wird beschlossen, dass für den 3D-Drucker vorerst kein Geld bereitgestellt
wird, unter anderem da noch keine Schätzung existiert. Die Bereitschaft der
Mitglieder zeigt außerdem, dass sich ein Crowdfunding-Ansatz lohnen würde, falls
dieser keinen Erfolg zeigt, kann Zuschuss aus Vereinsmitteln beantragt werden.

\subsection{Bericht T-Shirts}
\consensus{T-Shirt-Kasse ca. ausbalanciert}
Neo Bechstein berichtet im Chat, dass die Einnahmen für die T-Shirts, deren
Bestellung auf der Vorstandssitzung am 4. März 2012 beschlossen wurde, etwa
unseren Ausgaben entsprechen. Viele Mitglieder haben erstaunlich viel Geld
dafür gespendet, andere wiederum kaum etwas.

%%%%%%%%%%%%
%% TOP  3 %%
%%%%%%%%%%%%
\section{Gemeinnützigkeit}
\consensus{Gemeinnützig\-keit sinnvoll, aber saubere Finanzunter\-lagen nötig}
Durch Abwesenheit des Schatzmeisters hat der Vorstand keinen genauen Überblick
über die Finanzen. Prinzipiell wird von allen anwesenden Mitgliedern die
Gemeinnützigkeit des Vereins als sinnvoll erachtet, es besteht aber auch
Konsens, dass für die Beantragung der Gemeinnützigkkeit auf jeden Fall die
Finanzen klar durchschaubar und sauber
aufgezeichnet werden müssen.

%%%%%%%%%%%%
%% TOP  4 %%
%%%%%%%%%%%%
\section{Bank wechseln?}
\novote
Da der Verein im Moment für jede ausgehende Überweisung 1{,}50€ und für
jede eingehende Überweisung 0{,}05€ an die kontoführende Norddeutsche Landesbank
zahlt, wird darüber diskutiert, die Bank zu wechseln oder günstigere Konditionen
bei der Norddeutschen Landesbank zu fordern. Auch das Kommunikationsverhalten
der uns betreuuenden Bankmitarbeiter lässt zu wünschen übrig. Neo Bechstein
schlägt auf Fernanfrage vor, dieses Problem erst nach der Anerkennung der
Gemeinnützigkeit anzugehen, und vorerst die Anzahl der ausgehenden Überweisungen
gering zu halten. Valodim schließt sich dem an und meint, dass das Thema bisher
keine große Priorität besitzt. Falls sich ein günstigeres Gebot auftut, kann man
darüber nachdenken, aber im Moment sollte man nichts forcieren.

Die anwesenden Vorstandsmitglieder bringen keine gegensätzliche Meinung vor.
Es gibt keinen weiteren Beschluss in dieser Sache.

%%%%%%%%%%%%
%% TOP  5 %%
%%%%%%%%%%%%
\section{Versicherung}
\consensus{Kein Bedarf bzw. mehr Informationen nötig}
Valodim hatte im Vorfeld über Versicherungspflichten von Vereinen recherchiert,
ist aber im Internet auf keine besonderen Pflichten gestoßen. Manche
Versicherungen bieten zudem nur Tarife für Privatanwender und Firmen an, die
Frage ist, inwiefern die Konditionen auf Verein übertragbar sind. Außerdem hat
er bei der Öffentlichen und bei VHV eine Anfrage gestellt, es gab aber noch
keine Rückmeldung.

\subsection{Hausratsversicherung}
Firmenangebote scheinen hier zu überdimensioniert zu sein. Jedoch werden für
Privattarife insbesondere auch nachts bewohnte Räumlichkeiten gefordert, was für
uns nicht zutrifft. Valodim und m00lean würden sich darum kümmern und ein
lokales Beratungsbüro aufsuchen, aber erstmal eine Antwort auf o.~g. Anfrage
abwarten.

\subsection{Haftpflichtversicherung}
In diesem Fall haftet die private Versicherung des jeweiligen Verursachers.
Vereine an sich verursachen keinen Sachschaden.\footnote{Zu klären aber:
was ist z.~B. bei Unfällen mit Schaden an Vereinseigentum, wie
Überflutung/Natur\-katastrophen/höhere Gewalt?
\emph{(Anm.~d. Protokollanten)}}

\subsection{Feuerschutzversicherung}
Die Hausratsversicherung haftet für beschädigtes Eigentum (bzw. nicht, wenn es
keine gibt), Gebäudeversicherung des Vermieters haftet für notwendige
Renovierungsarbeiten. Die Frage ist, was in diesem Fall der Mietvertrag regelt.

Außerdem sollte man sich um einen Feuerlöscher und einen Rauchmelder kümmern.
rohieb stellt sich zur Verfügung, dazu nähere Informationen einzuholen.

\subsection{Rechtsschutzversicherung}
Valodim ist der Meinung, dass sich eine Rechtsschutzversicherung nicht lohnt.
Außerdem stellt sich die Frage, wer einen "`Hackerspace"' zu günstigen
Konditionen versichern will.

Generell scheint es auch sinnvoll, sich bei anderen Hackerspaces über dieses
Thema umzuhören.\footnote{Andere Hackerspaces scheinen dies aber in ähnlicher,
minimalistischer Hinsicht zu behandeln\ldots \emph{(Anm.~d. Protokollanten)}}

%%%%%%%%%%%%
%% TOP  7 %%
%%%%%%%%%%%%
\section{Vorgehen bei Partys/Veranstaltungen}
\vote{Pfand 50€ für Partys, Erstattung bei Sauberkeit bis 20:00 am nächsten
Tag}{5}{0}{0}
Nachdem bei vorherigen Veranstaltungen der Hackerspace für nächsten Tag in einem
sehr unsauberen Zustand hinterlassen wurde, kam der Vorschlag auf, ein Pfand für
solche Veranstaltungen einzuführen, das für die Nutzung des Hackerspaces zu
entrichten ist. Bei erfolgreicher Reinigung bis 20:00 am Tag nach der
Veranstaltung wird das Pfand zurückgezahlt. Falls die Reinigung in diesem
Zeitraum nicht stattfindet, wird das Pfand an die Entität ausgezahlt, die
stattdessen für Ordnung sorgt.

Dieser Vorschlag fand auf der Mailingliste relativ viel Zuspruch und wird auch
von den anwesenden Vorstandsmitgliedern ohne Enthaltung und ohne Gegenstimmen
angenommen.

%%%%%%%%%%%%
%% TOP  8 %%
%%%%%%%%%%%%
\section{Neue T-Shirts/Pullover}
\consensus{Bedarf an Bekleidung abwarten, Vorrat an Größen behalten}
Von der letzten Bestellung sind bisher nur noch übrig geblieben:
\begin{itemize}
	\item 4 T-Shirts, unisex, Größe XL
	\item 1 T-Shirt, unisex, Größe XXL
\end{itemize}

Als Konsens wird erst einmal beschlossen, auf weiteren Bedarf zu warten%
\footnote{siehe Wiki: \url{https://stratum0.org/wiki/Bekleidung}}, und dann
eine weitere Sammelbestellung aufzugeben. Außerdem sollen von den am meisten
nachgefragten Größen etwa 2 Exemplare auf Vorrat gehalten werden. Auch die
Bestellung von Pullovern erscheint allen Anwensenden sinnvoll.

%%%%%%%%%%%%
%% TOP  9 %%
%%%%%%%%%%%%
\section{Sonstiges}
\subsection{Beamer}
\consensus{Geld und Angebote abwarten}
Zur Zeit hat der Verein nicht genug Geld über, um einen Beamer anzuschaffen, der
für unsere Bedürfnisse ausreichen würde. Valodim schlägt vor, etwa ein halbes
Jahr zu warten und dann noch einmal die finanzielle Lage zu überprüfen, oder
eine Pledgefunding-Aktion zu starten. Die Vorstandsmitglieder schließen sich
dieser Meinung an.

\subsection{VPN}
\consensus{VPN gerne, wer macht hat Recht.}
Drahflow hatte angeregt, ein Virtual Private Network mit Endpunkt im
Stratum 0 einzurichten, um Mitgliedern an Orten mit beschränktem Internetzugang
trotzdem einen freien Netzzugang zu gewähren. Der Vorstand ist diesem Vorschlag
gegenüber positiv gesinnt, jedoch soll kein bloßer SSH-Zugriff bestehen, da
die Gefahr besteht, dass private Serverdienste in den Hackerspace ausgelagert
werden und somit den Upstream der DSL-Leitung verstopfen. Auf der Mailingliste
zeigte sich zu diesem Vorschlag auch keine negative Resonanz. Somit muss nur
noch eine Entität gefunden werden, die sich bereit erklärt, ein solches System
einzurichten.

\begin{description}
	\item[Sitzung geschlossen] um 18:46
\end{description}
\end{document}
% vim: set tw=80 et sw=2 ts=2:
