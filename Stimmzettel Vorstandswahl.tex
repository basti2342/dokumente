\documentclass[a5paper,12pt]{scrartcl}
\usepackage[utf8]{inputenc}
\usepackage[T1]{fontenc}
\usepackage[ngerman]{babel}
\usepackage{tabularx}
\usepackage{libertine} % kann man notfalls auch ignorieren, wenns nicht da ist
%\usepackage{textcomp} % für €
%\usepackage[cm]{fullpage}

\newenvironment{kandidaten}[1]{
  \subsection*{#1}
  \tabularx{\textwidth}{|l|X|}
    \hline
}{
  \endtabularx
}
\newcommand{\wahltabellenzeile}[1]{#1 \rule[-3pt]{0pt}{16pt}\phantom{asdf} & \\ \hline}
\newcommand{\wahltabelle}[1]{
  \begin{kandidaten}{#1}
    \wahltabellenzeile{Kandidat 1}
    \wahltabellenzeile{Kandidat 2}
    \wahltabellenzeile{Kandidat 3}
  \end{kandidaten}
}

\pagestyle{empty}
\usepackage[top=1cm, bottom=1cm]{geometry}
%\addtolength{\pageheight}{60pt}
%\setlength{\voffset}{-12pt}

\begin{document}
\begin{center}
  {\Large\bfseries Vorstandswahl Stratum~0~e.~V. \\[0.5em] Stimmzettel}
  \\[1em]
  {\large 8. Januar 2012}
\end{center}


\wahltabelle{1. Vorsitzender}
\wahltabelle{stellvertretender Vorsitzender}
\wahltabelle{Kassenwart}
\begin{kandidaten}{Beisitzer}
  \wahltabellenzeile{Kandidat 1}
  \wahltabellenzeile{Kandidat 2}
  \wahltabellenzeile{Kandidat 3}
  \wahltabellenzeile{Kandidat 4}
  \wahltabellenzeile{Kandidat 5}
  \wahltabellenzeile{Kandidat 6}
\end{kandidaten}

\end{document}
% vim: set tw=80 et sw=2 ts=2:
