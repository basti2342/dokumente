\documentclass[a4paper,12pt]{scrartcl}
\usepackage[utf8]{inputenc}
\usepackage[T1]{fontenc}
\usepackage[ngerman]{babel}
\usepackage{libertine} % kann man notfalls auch ignorieren, wenns nicht da ist
\usepackage{textcomp} % notfalls für €

\renewcommand{\labelenumi}{(\arabic{enumi})}
\renewcommand{\labelitemi}{--}
\makeatletter\renewcommand*\thesection{TOP\ \@arabic\c@section}\makeatother
\setcounter{section}{-1}

\title{Vorstandssitzung des Stratum~0~e.~V.}
\date{11. Dezember 2011}

\begin{document}
\maketitle

\begin{description}
 \item[Anwesend:] Valodim (Vorsitzender) \\
    n30\_83c45731n (Schatzmeister, verspätet) \\
    rohieb (Beisitzer) \\
    Ortwin (Beisitzer) \\
    larsan (Beisitzer)
  \item[Entschuldigt:] m00lean (stv.~Vorsitzender)
  \item[Veranstaltung eröffnet] durch Valodim um 14:38
  \item[Protokoll:] rohieb
\end{description}

\section{Mitgliedsanträge}
Folgender Mitgliedsantrag wurden mit Wirkung zum Datum der Antragstellung 
einstimmig (mit Ausnahme des verspäteten n30\_83c45731n) angenommen:
\begin{itemize}
 \item Benjamin Klein, zum 3.~Oktober~2011
\end{itemize}

\emph{[n30\_83c45731n erscheint verspätet um 14:43]}

n30\_83c45731n merkt zu der Entscheidung an, dass Benjamin Klein seinen Antrag
auf Ermäßigung des Mitgliedsbeitrags zurückgezogen hat. Siehe dazu auch 
\ref{top:nachweis-vs-vertrauen}.

\section{Konto und Finanzen}
\label{top:finanzen}
\subsection{Bericht des Vorsitzenden}
Valodim hat den ersten Kontoauszug mit Datum vom 28.~November von der Bank 
erhalten. Der aktuelle Kontostand beläuft sich auf $693{,}17$€. Bei Lichte 
betrachtet ergibt dies durchschnittliche Einnahmen von 115€ pro Monat (bei 
6 Monaten) und $3{,}39$€ monatliche Beiträge pro Mitglied (bei 34 Mitgliedern), 
was insofern eine eher magere Ausbeute darstellt.

\subsection{Beiträge}
Valodim schlägt auch vor, das Beitragsmodell zu ändern, bis eine Räumlichkeit
gefunden ist, da ein Beitrag von 20€ nicht mit dem aktuellen Stand der Dinge 
vereinbar ist. Es werden folgende Vorschläge gemacht:
\begin{itemize}
  \item Den bisherigen Beitrag auf 10€ voll/6€ ermäßigt halbieren, bisherige
    Beiträge als Puffer für Kaution, Provision etc. behalten. Valodim geht davon
    aus, dass bei den Mitgliedern eine gewisse Zahlungsbereitschaft besteht; und
    6€ für Studenten sollte im vertretbaren Rahmen liegen. Eine ermäßigten
    Beitrag findet er nicht verkehrt.    
  \item Alle Mitglieder zahlen gleich viel (z.~b. 10€). n30\_83c45731n regt an,
    dass dies den Verwaltungsaufwand niedrig hält.
  \item Zusätzlich wird angeregt, die bisherigen angefallenen, aber noch 
    nicht gezahlten Beiträge pauschal auf die Hälfte der zu zahlenden 
    Beiträge zu deckeln. Aus dem Chat kommt an dieser Stelle die Frage, wie
    mit Mitgliedern verfahren wird, die schon Beiträge gezahlt haben. Larsan
    schlägt vor, dass in diesem Fall die gezahlten Beiträge angerechnet 
    werden. Ortwin wirft an dieser Stelle ein, dass das Modell möglichst fair 
    und unkompliziert gestaltet sein muss.
\end{itemize}

Es wird vorerst keine favorisierte Lösung gefunden. In jedem Fall muss sowieso
eine Mitgliederversammlung über das weitere Verfahren abstimmen.

Von n30\_83c45731n kommt die Idee, nach einem Jahr (also im Juli 2012) jedem 
Mitglied eine persönliche Bilanz der gezahlten Beiträge zukommen zu lassen. 
Valodim und rohieb opponieren dagegen und zweifeln an der Nützlichkeit. 
Wenn überhaupt, sollte dies eher früher (und häufiger) als später passieren.

Ein Großteil der Mitglieder spricht sich angesichts der bisher spärlichen 
Einkünfte dafür aus, dass ausstehende Beiträge aktiv eingetrieben werden 
sollten, um den Verein handlungsfähig zu halten. n30\_83c45731n spricht sich 
allerdings dagegen aus und vertritt einen humanistischen Ansatz.

\subsection{Kontoführungsgebühr}
Im Moment wird uns von der Sparkasse jede eingehende Überweisung mit $0{,}05$€ 
in Rechnung gestellt, was ein Sicherheitsloch bei kleinwertigen 
Massenüberweisungen darstellen könnte. Valodim schlägt vor, bei anderen Banken
nach Konditionen diesbezüglich anzufragen, und notfalls die Bank zu wechseln; 
und bei der Sparkasse auf den Umstand hinzuweisen, falls sich bei anderen Banken
bessere Konditionen ergeben. Das Problem wird aber allgemein vorerst eher als
niedrig priorisiert eingeschätzt.

\section{Bericht Raum-Situation}
\label{top:raum}
Kurz: die Raumsituation ist beschissen. Der Ansatz "`Wenn wir nicht zum Raum 
kommen, kommt der Raum zu uns"' (Makler fragen, etc.) funktioniert anscheinend
in der Realität nicht. Einerseits ist ein ordentlicher Raum erst mit genügend 
(zahlenden) Mitgliedern bezahlbar, andererseits haben mehrere Leute angekündigt,
beizutreten, falls es einen Raum gäbe. Es gibt verschiedene Vorschläge, diesem 
Teufelskreis zu entfliehen sowie die Mitglieder bei der Stange zu halten und den
Verein für neue Mitglieder attraktiv zu machen:
\begin{itemize}
  \item Valodim regt an, als Zwischenlösung einen kleineren Raum zu mieten
    (siehe Realitätscheck-Mail auf der Mailingliste)
  \item rohieb schlägt vorerst regelmäßige (wöchentliche?) Stammtische/Treffen 
    vor, bis eine geeignete Räumlichkeit gefunden ist. Hier würden sich z.~B.
    das Monkey Island im Affenfelsen anbieten, welches vom MonkeyRock~e.~V.
    betrieben wird. Allerdings ist eine Anfrage n30\_83c45731ns dorthin bisher 
    im Sande verlaufen.
  \item n30\_83c45731n nennt in diesem Zusammenhang auch den Bürgerverein 
    Weststadt, der Räumlichkeiten anbietet, und erklärt sich bereit, den Kontakt
    dorthin herzustellen.
  \item Es wird auch das Jugendzentrum Neustadtmühle als Treffpunkt 
    vorgeschlagen, allerdings wirft Valodim ein, dass unsere Zielgruppe doch
    eher erwachsene Personen sind.
  \item Schließlich bietet sich noch die Anwerbung von Sponsoren als Option an
    (siehe \ref{top:sponsoren}), um die aktuelle Finanzsituation zu 
    entschärfen und eine adäquate Räumlichkeit anzumieten.
\end{itemize}

In jedem Fall sollte ein Meinungsbild der Mitglieder zu dieser Thematik 
eingeholt werden. Als monatliche Obergrenze für die Raummiete im Falle von 
regelmäßigen Treffen einigen sich die Anwesenden auf 80€. Auch wird angemerkt,
dass ein Termin für ein regelmäßiges Treffen erst nach Feststehen des 
Treffpunktes festgelegt werden kann.

Für den Fall einer eigenständigen Räumlichkeit einigen sich die Anwesenden auf
folgende Eckdaten:
\begin{itemize}
  \item vorerst 30-40~qm, mit Option auf Umzug
  \item 500€ inkl. Strom, Wasser, Heizung als harte Obergrenze
  \item möglichst vor der Besichtigung klarmachen, wer wir sind und was wir tun
\end{itemize}

Außerdem schlägt Valodim vor, regelmäßige Treffen bei ihm zur Spacesuche 
abzuhalten, um den Immobilienmarkt im Auge zu behalten. rohieb bietet sich an,
eine entsprechende Mail an die öffentliche Mailingliste zu formulieren.

\section{Sponsoren}
\label{top:sponsoren}
Drahflow hatte vor einiger Zeit angeregt, pizza.de als Sponsor zu werben.
n30\_83c45731n erklärt sich bereit, eine offizielle Anfrage zu stellen.

Auf der Sitzung vom 30.~August regte m00lean an, den Stammtisch der Gruppe 
"`IT-Region 38"' zur Sponsorenwerbung zu nutzen. Es herrscht Einigkeit darüber,
diese Idee weiter zu verfolgen.

\section{Mitgliederversammlung/Weihnachtsfeier}
n30\_83c45731n hatte vor einiger Zeit vorgeschlagen, eine Weihnachtsfeier im
Monkey Island zu organisieren. Diese sollte allerdings wegen der akuten 
Terminlage nicht im Dezember, sondern im Januar stattfinden. Valodim hat 
Kontakte zum Vorstand des Monkey Island und erklärt sich bereit, dort 
anzufragen; als Alternative könnte auch die Schuntille im Studentenwohnheim an 
der Schunter genutzt werden. Als Termin wird vorerst das dritte 
Januarwochenende angepeilt. Die Anwesenden einigen sich auf maximale Kosten 
von 150€ für die Feier.

In Verbindung mit der Weihnachtsfeier sollte auch eine Mitgliederversammlung
abgehalten werden, um über das Logo zu entscheiden, sowie über Raumsituation und
regelmäßige Treffen (siehe \ref{top:raum}) sowie Beiträge (siehe 
\ref{top:finanzen}) zu beraten.

n30\_83c45731n wird die Mitglieder einladen, da er alle nötigen Daten hat.

\section{Werbung}
Werbung wird verschoben auf einen Zeitpunkt, an dem das Logo feststeht.

\section{Sonstiges}
\label{top:nachweis-vs-vertrauen}
Von rohieb kam die Anregung, die Politik in Bezug auf Beitragsermäßigung 
festzulegen, da im Moment von Seiten des Schatzmeisters der ermäßigte Beitrag 
eher Vertrauensbasis gewährt wird. n30\_83c45731n entgegnet, dass ein Nachweis
zur Ermäßigung zwar in der Beitragsordnung gefordert wird, aber dass der
Nachweis für Arbeitslosengeld-Empfänger sehr viele persönliche Daten enthält, 
die eigentlich einen Verein nichts angehen sollten. Insofern sollte man den 
Datenschutz berücksichtigen.

Die Anwesenden kommen zum Kompromiss, dass zumindest bei Studenten der Nachweis 
durch eine Immatrikulationsbescheinigung keinen großen Aufwand darstellt und 
der Datenschutz gewährleistet ist. Bei Arbeitslosengeld-Empfängern wird im 
Einzelfall entschieden, ob ein Nachweis gefordert wird.


\begin{description}
\item[Veranstaltung geschlossen] durch den Vorsitzenden um 16:57.
\end{description}

\end{document}
