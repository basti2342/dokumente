\documentclass[a4paper,12pt]{scrartcl}
\usepackage[utf8]{inputenc}
\usepackage[T1]{fontenc}
\usepackage[ngerman]{babel}
\usepackage{libertine} % kann man notfalls auch ignorieren, wenns nicht da ist
\usepackage{textcomp} % für €
\usepackage[transcript]{stratum0doc}
\usepackage[colorlinks=false]{hyperref}

\title{Vorstandssitzung des Stratum~0~e.~V.}
\date{11.~Oktober~2012}

\begin{document}
\maketitle
\tableofcontents

\begin{description}
 \item[Anwesend:] Valodim (Vorsitzender) \\
    Neo Bechstein (Schatzmeister) \\
    rohieb (Beisitzer) \\
    larsan (Beisitzer)
  \item[Sitzung eröffnet] um 11:32
  \item[Versammlungsleitung:] Valodim (Wahl durch Handzeichen)
  \item[Protokoll:] rohieb (Wahl durch Handzeichen)
\end{description}

%%%%%%%%%%%%
%% TOP  0 %%
%%%%%%%%%%%%
\section{Anträge und Umlaufbeschlüsse}
\subsection{Mitgliedsanträge}
\vote{Beitritt Chris Fiege}{4}{0}{0}
\vote{Beitritt Stephan Merker}{4}{0}{0}
Die anwesenden Vorstandsmitglieder beschließen einstimmig, folgende Entitäten
als Mitglieder im Stratum~0~e.~V. aufzunehmen:
\begin{itemize}
	\item Chris Fiege, zum 18. Juni 2012
	\item Stephan Merker, zum 29. Juni 2012
\end{itemize}

\subsection{Umlaufbeschlüsse}
Die anwesenden Vorstandsmitglieder bestätigen einstimmig folgende
Umlaufbeschlüsse:
\vote{250€ für Küchen\-einrichtung}{4}{0}{0}
\vote{4 weitere Schlüssel}{4}{0}{0}
\begin{itemize}
	\item Umlaufbeschluss vom 1. Oktober 2012: 250€ für eine neue, komplette
		Kücheneinrichtung, inkl. Boiler oder Durchlauferhitzer. Das Plenum hatte
		diesem Antrag zugestimmt.
	\item Umlaufbeschluss vom 8. Oktober 2012: 4 weitere Schlüssel für die obere
		Tür nachmachen lassen, sodass insgesamt 10 Schlüssel existieren. Die
		Schlüssel werden vorerst bei Valodim verwahrt und nach Bedarf an Mitglieder
		vergeben.
\end{itemize}

\subsection{Antrag: 80€ für Spacelight}
\vote{80€ Projektgeld für Spacelight}{4}{0}{0}
DooMMasteR hatte den Antrag gestellt, 80€ für das Projekt "`Spacelight"' zu
bewilligen. Das Plenum hatte diesem Antrag zugestimmt. Die anwesenden
Vorstandsmitglieder stimmen dem Antrag einstimmig zu.

\subsection{Antrag: 96€ für Projektboxen, \textsc{LACK}-Tische, Klappstühle}
\vote{96€ für Projektboxen, \textsc{LACK}-Tische, Klappstühle}{4}{0}{0}
chrissi\textasciicircum{} hatte den Antrag gestellt, 96€ für
\textsc{IKEA}-Einkäufe auszugeben. Insbesondere ist der Vorrat an freien
Aufbewahrungsboxen für Projekte zurückgegangen. Des weiteren sollen Klappstühle
angeschafft werden, um genug Sitzgelegenheiten bei Vorträgen bieten zu können.
Außerdem sollen noch zwei kleine Tische für den Sofaraum beschafft werden.
Das Plenum hatte diesem Antrag zugestimmt, auch die anwesenden
Vorstandsmitglieder stimmen dem Antrag einstimmig zu.

%%%%%%%%%%%%
%% TOP  1 %%
%%%%%%%%%%%%
\section{Feuerlöscher}
\vote{max. 80€, ein Pulverlöscher, ein CO$_2$-Löscher}{4}{0}{0}
Aus Sicherheitsgründen sollte im Space mindestens ein Feuerlöscher vorrätig
sein. Es wird darüber diskutiert, ob hier Pulverlöscher oder CO$_2$-Löscher
sinnvoller sind. Pulverlöscher sind zwar ergiebiger, aber die
Hinterlassenschaften sind weniger gut zu säubern. CO$_2$-Löscher dagegen
könnten für umliegende Elektronikgeräte schonender sein. Es wird einstimmig
beschlossen, je einen Pulverlöscher und einen CO$_2$-Löscher anzuschaffen,
die zusammen insgesamt maximal 80€ kosten sollen.

%%%%%%%%%%%%
%% TOP  2 %%
%%%%%%%%%%%%
\section{Stromzähler}
\consensus{Fachkraft für Stromzähler\-einbau finden, oder Haus\-meister nach
Zwischen\-zähler fragen}
Der Stromzähler, der von einem Mitglied beschafft wurde, sollte mal eingebaut
werden, dazu ist eine Fachkraft mit entsprechender Ausbildung nötig. Falls der
Einbau auf lange Sicht nicht klappt, sollte beim Hausmeister ein Zwischenzähler
angefragt werden. Valodim erklärt sich dazu bereit, mit dem Hausmeister zu
reden. Einen Fachmann kann man sicher über die Mailingliste ausfindig
machen.\footnote{Der Zähler wurde im November 2012 von einem Fachmann
eingebaut.}

%%%%%%%%%%%%
%% TOP  3 %%
%%%%%%%%%%%%
\section{Versicherung}
\consensus{Valodim kümmert sich drum}
Valodim hat sich in Bezug auf Versicherungen bei anderen Hackerspaces umgehört.
Einige Versicherungen wie Inhaltsversicherung und Haftpflicht scheint sinnvoll
zu sein. Er wird sich in dieser Hinsicht mit dem Schatzmeister zusammensetzen
und Beratung einholen.

%%%%%%%%%%%%
%% TOP  4 %%
%%%%%%%%%%%%
\section{Mitgliederversammlung}
\consensus{rohieb kümmert sich drum}
Es steht demnächst wieder eine Mitgliederversammlung an, da die Amtszeit des
gewählten Vorstands Anfang Januar 2013 abläuft. Es wird ein Termin in der ersten
Dezemberhälfte angepeilt. rohieb erklärt sich bereit, eine Umfrage nach dem
genauen Termin vorzubereiten und sich um mögliche Räumlichkeiten und weitere
Formalien zu kümmern.

Als weiterer Punkt wird angesprochen, ob man dem Vorstand durch die
Mitgliederversammlung geben lässt. Insbesondere die Miete einer neuen
Räumlichkeit erfordert im Moment eine Mitgliederversammlung, die aber mit 2
Wochen Vorlaufzeit unter Umständen zu lange braucht, um zeitkritische Angebote
im aktuell sehr fluktuierenden Immobilienmarkt abzusegnen.\footnote{Das Thema
wurde allerdings auf der Mitgliederversammlung am 15. Dezember 2012 nicht
behandelt.}

%%%%%%%%%%%%
%% TOP  4 %%
%%%%%%%%%%%%
\section{Sonstiges}
\subsection{Bankgebühren}
DooMMasteR hatte erwähnt, dass es bei der Sparkasse Gifhorn-Wolfsburg günstigere
Konditionen für Vereine gibt. Er wollte sich in der Hinsicht nochmal ein genaues
Angebot geben lassen.

\subsection{Zahlungsmoral der Mitglieder}
\consensus{Neo erinnert nochmal entsprechende Mitglieder, nachträgliche
Ermäßigung ist möglich!}
Es gibt immer noch mehrere Mitglieder, die nur sporadisch ihren Mitgliedsbeitrag
zahlen. Ein genereller Erlass scheint nicht sinnvoll zu sein, da laufende Kosten
bezahlt werden müssen. Es erscheint also höchstens sinnvoll, offene Beträge zu
erlassen, die vor dem Zeitpunkt der Space-Anmietung aufgelaufen sind. Außerdem
sollte man nochmal darauf hinweisen, dass die Ermäßigung auch rückwirkend
möglich ist.  Generell sollte man in der Hinsicht den Mitgliedern entgegen
kommen und eine individuelle Lösung finden, die allen Parteien passt. Das Thema
soll außerdem auf der nächsten Mitgliederversammlung diskutiert werden.

Neo erklärt sich bereit, nochmals die Mitglieder anzuschreiben, die im Rückstand
sind, die Formulierung kann gemeinsam erarbeitet werden.

\consensus{rohieb macht Mitgliedsanträge in Papierform}
rohieb regt noch an, dass er Mitgliedsanträge in Papierform entwerfen will, die
im Space ausgelegt werden können und auch eine gewisse Form der Werbung sind.
Mitgliedsanträge in Papierform entsprechen außerdem eher der Gewohnheit der
Masse, nicht alle Mitglieder wissen, dass bei uns formlose E-Mails genügen.

\begin{description}
	\item[Sitzung geschlossen] um 13:04
\end{description}
\end{document}
