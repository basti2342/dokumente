% vim:tw=80 ts=2 et sw=2 indentexpr= :
\documentclass[a4paper,12pt]{scrartcl}
\usepackage[utf8]{inputenc}
\usepackage[T1]{fontenc}
\usepackage[ngerman]{babel}
\usepackage{libertine} % kann man notfalls auch ignorieren, wenns nicht da ist
\usepackage{textcomp} % notfalls für €
\usepackage{lastpage}
\usepackage{fancyhdr}
\usepackage[colorlinks=false]{hyperref}

\renewcommand{\labelenumi}{(\arabic{enumi})}
\renewcommand{\labelitemi}{--}
\makeatletter\renewcommand*\thesection{TOP\ \@arabic\c@section}\makeatother
\newcommand{\abstimmung}[4]{\marginpar{\footnotesize #1:\\#2~pro, #3~con, %
  #4~neutral}}
\setcounter{section}{-1}
\addtolength{\textwidth}{-10pt}
\addtolength{\marginparwidth}{10pt}

\newcommand{\mytitle}{Mitgliederversammlung des Stratum~0~e.~V.}
\newcommand{\mydate}{8. Januar 2012}

\title{\mytitle}
\date{\mydate}
\pagestyle{fancy}
\fancyhf{}
\chead{\mytitle{} -- \mydate}
\cfoot{Seite \thepage\ von \pageref{LastPage}}
\fancypagestyle{plain}{%
  \renewcommand{\headrulewidth}{0pt}%
  \fancyhf{}%
  \fancyfoot[C]{Seite \thepage\ von \pageref{LastPage}}%
}
\begin{document}
\maketitle

%%%%%%%%%%%
%% TOP 0 %%
%%%%%%%%%%%
\section{Eröffnung}
\begin{description}
  \item[Zeit:] 9. Januar 2012, 15:00
  \item[Ort:] beyti Grillhaus, Bohlweg; später Plaza des Informatikzentrums,
    Mühlenpfordtstraße
  \item[Anwesend:] 28 akkreditierte Mitglieder, davon 27 mit Stimmrecht
  \item[Wahl des Versammlungsleiters:] Valodim durch Handzeichen; nimmt die 
    Wahl an
  \item[Protokoll:] rohieb erklärt sich bereit, keine Gegenstimmen
  \item[Veranstaltung eröffnet] durch den Versammlungsleiter um 15:50
  \item[Tagesordnung:] ohne Gegenstimmen und Enthaltungen angenommen
\end{description}

%%%%%%%%%%%
%% TOP 1 %%
%%%%%%%%%%%
\section{Berichte}
\subsection{Bericht des Vorstandes}
Valodim stellt zuerst den neu hinzugekommenen Mitglieder die amtierenden
Vorstandsmitglieder vor. Darauf fasst er die Aktivitäten des Vorstands im
vergangenen Jahr zusammen:
\begin{description}
  \item[30. August 2010:] 1. Vorstandssitzung.\footnote{Vollständiges
    Protokoll: \url{https://stratum0.org/index.php/Vorstandssitzung_2011-08-30}}
    Themen: Überblick über Immobiliensituation, Besichtigungen, Eintragung,
    Konto.
  \item[24. September 2012:] positive Rückmeldung vom Amtsgericht über die
    Eintragung ins Vereinsregister
  \item[4. Oktober 2010:] Eröffnung eines Vereinskontos bei der
    Braunschweigischen Landessparkasse, verlief nicht ganz reibungslos aufgrund
    der laufenden EDV-Umstellung der Bank
  \item[30. November 2010:] Erster Kontoauszug
  \item[11. Dezember 2010:] 2. Vorstandssitzung.\footnote{Vollständiges
    Protokoll: \url{https://stratum0.org/index.php/Vorstandssitzung_2011-12-11}}
    Themen: Finanzen, Bericht Raum-Situation, Mitgliederversammlung, 
    Weihnachtsfeier
  \item[27.-20. Dezember 2010:] Streaming der Vorträge vom 28. Chaos
    Communication Congress im Rahmen des Programms "`No Nerd Left Behind"', in
    Zusammenarbeit mit der Fachgruppe Informatik der TU Braunschweig. Es waren
    auch viele Nicht-Studenten in den Räumen der TU anzutreffen.
\end{description}

Die Suche nach Räumlichkeiten lief nebenher und stagnierte teilweise, da der 
Immobilienmarkt kaum veränderlich war. Darum wurden auf der 2. Vorstandssitzung
wöchentliche Treffen zur Beobachtung des Immobilienmarktes angeregt, von denen
das erste Treffen am 18. Dezember auch gleich zur Besichtigung der Räumlichkeit
an der Hamburger Straße am 20. Dezember bzw. in wiederholter Form am 3. Januar
führte (siehe \ref{top:space}).

Gleichzeitig waren während des Jahres im Rahmen der Aktion "`Zeitabgleich"'
mehrere Mitglieder des Vereins in verschiedenen Hackerspaces in Deutschland zu
Gast und haben Werbung für unseren Verein gemacht: Neo Bechstein, ktrask und
m00lean Anfang September beim \emph{chaosdorf} in Düsseldorf zum Kongress 
OpenRheinRuhr, whisp in der \emph{c-base} in Berlin, mehrere Mitglieder in 
Hannover bei der \emph{leitstelle511} Anfang August, und blinry live auf dem 
28. Chaos Communication Congress in Berlin Ende Dezember.

Außerdem wurde auf der 1. Vorstandssitzung der Beschluss getroffen, zuerst keine
Gemeinnützigkeit beim Finanzamt zu beantragen, um die Rückabwicklung der
Mitgliedsbeiträge zum Wohle der Mitglieder im Falle eines Scheiterns des
Vereins einfacher zu gestalten. Andernfalls könnten die eingenommen Beiträge nur
einem weiteren, gemeinnützigen Verein zu Gute kommen. Grundsätzlich wird der
Status der Gemeinnützigkeit aber weiterhin angestrebt, sobald der Verein stabil
läuft und eine Räumlichkeit gefunden ist. Aus Valodims Sicht wird aber dieser
Grund gegen die Gemeinnützigkeit aktuell im Laufe der Zeit immer weiter
hinfällig. Außerdem ist die Gemeinnützigkeit zwar in der Satzung festgehalten,
aber sie interessiert kaum jemanden, solange der Verein keine
Steuervergünstigungen beim Finanzamt beantragt.

\section{Bericht des Schatzmeisters und der Kassenprüfer}
Wie schon erwähnt hat sich die Einrichtung des Kontos etwas verzögert, wodurch
erste Überweisungen von Mitgliedern nicht ausgeführt werden konnten. Dies ist
aber inzwischen nach einem nochmaligen Gespräch des Vorstandes mit der Bank
behoben. Außerdem hat sich gezeigt, dass uns pro eingehender Überweisung 5 Cent
Kontoführungsgebühr angerechnet werden, worüber der Vorstand bei der Eröffnung
des Kontos nicht aufgeklärt wurde. Der Schatzmeister wird diese Situation noch
zeitnah mit der Bank besprechen; es besteht wohl diesbezüglich
Verhandlungsbereitschaft seitens der Bank.

An Mitgliedsbeiträgen sind seit Gründung des Vereins im Juni 2010 bis zum 
Zeitpunkt der letzten Kontenübersicht einige Tag vor der Versammlung bisher 
insgesamt 2.088€ überwiesen worden. Der Schatzmeister weist aber auch scharf
darauf hin, dass zu diesem Zeitpunkt noch 1.490€ an offenen Beiträgen
ausstanden. Davon wurde wiederum vor Beginn der Versammlung ein Großteil durch
Barzahlung ausgeglichen, sodass nun noch etwa 200€ an Forderungen von
Mitgliedern offen sind. Außerdem sind dem Verein 252{,}79€ an Spendengelder
zugekommen.

Auf der anderen Seite sind als Ausgaben aufgelaufen:
\begin{itemize}
  \item 26{,}78€ Notarkosten für die Eintragung,
  \item 8{,}90€ Kontoführungsgebühr
  \item außerdem noch etwa 70€ Bearbeitungskosten des Amtsgerichts für die
    Eintragung ins Vereinsregister, die bisher von Valodim ausgelegt wurden, und
    die er noch nicht vom Verein zurück erhalten hat.
\end{itemize}

Die Kassenprüfer konnten alle Kontenbewegungen nachvollziehen, bis auf den
genauen Betrag der Spenden, der sich in ihrer Rechnung minimal unterscheidet,
was wohl auf einen Übertragungsfehler seitens des Schatzmeisters zurückzuführen
ist. Das verbuchte Gesamtguthaben am Beginn der Veranstaltung beläuft sich 
demnach auf 3.320€.

Die Kassenprüfer merken noch an, dass die Aufzeichnungen des Schatzmeisters
etwas schwer nachzuvollziehen sind. Sie erklären sich aber bereit, zusammen
mit dem Schatzmeister eine nachvollziehbarere Methode zu finden.

\subsection{Entlastung des Vorstandes und des Schatzmeisters}
Terminar spricht hier den Punkt an, dass zwischendurch teilweise zu viel Stille
herrschte und der Vorstand nur schwer zu erreichen war. Er sieht es nicht als
grundsätzliches Problem an, wenn zwischendurch Funkstille herrscht, und es
nichts zu berichten gibt, nur sollten voraussehbarere Abwesenheitszeiten
angekündigt und eine Vertretung bestellt werden, sodass zumindest zu jedem
Zeitpunkt eine minimale Erreichbarkeit besteht. Anscheinend gab es zum selben
Zeitpunkt auch noch technische Probleme auf der öffentlichen Mailingliste,
m00lean wirft aber ein, dass diese schnell behoben wurden. Grundsätzlich
schließen sich aber Valodim und Neo Bechstein den Ausführungen von Terminar
an und sehen Optimierungsbedarf in solchen Fällen.

\abstimmung{Entlastung des Vorstands}{18}{0}{9}
Für die Entlastung des Vorstandes wird per Handzeichen abgestimmt. 18 Entitäten
stimmen dafür, 9 enthalten sich, es gibt keine Gegenstimmen. \\

\emph{[16:26 bis 17:05: Pause, Relokation auf die Plaza des Informatikzentrums]}






\begin{description}
\item[Veranstaltung geschlossen] durch den Versammlungsleiter um 20:02.
\end{description}

\end{document}
