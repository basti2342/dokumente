\documentclass[a4paper,12pt]{scrartcl}
\usepackage[utf8]{inputenc}
\usepackage[T1]{fontenc}
\usepackage[ngerman]{babel}
\usepackage{libertine} % kann man notfalls auch ignorieren, wenns nicht da ist
\usepackage{textcomp} % für €
\usepackage[transcript]{stratum0doc}
\usepackage[colorlinks=false]{hyperref}

\addtolength{\textwidth}{-10pt}
\addtolength{\marginparwidth}{10pt}

\title{Vorstandssitzung des Stratum~0~e.~V.}
\date{5.~Februar~2012}

\begin{document}
\maketitle

\begin{description}
 \item[Anwesend:] Valodim (Vorsitzender) \\
    m00lean (stv.~Vorsitzender) \\
    Neo Bechstein (Schatzmeister) \\
    rohieb (Beisitzer) \\
    larsan (Beisitzer) \\
    dstulle (Gast ohne Stimmrecht)
	\item[Entschuldigt:] Ortwin (Beisitzer)
  \item[Sitzung eröffnet] durch Valodim um 18:25
  \item[Protokoll:] rohieb
\end{description}

%%%%%%%%%%%%
%% TOP  0 %%
%%%%%%%%%%%%
\section{Anträge und Umlaufbeschlüsse}
\subsection{Bewilligung Kaution Schimmel-Hof von 900€}
\vote{Bewilligung Kaution}{5}{0}{0}
Die Kaution zur Miete des aktuellen Hackerspaces in der Hamburger Straße 273a
(Schimmel-Hof) in Höhe von 900€ in bar wird ohne Gegenstimmen und Enthaltungen
angenommen.

\subsection{Internetzugang von 1\&1}
\vote{Bestellung 1\&1 DSL}{5}{0}{0}
Für den Internetzugang wird beschlossen, den DSL-Tarif "`Doppel-Flat 50.000
Office"' von der Firma 1\&1 mit 50~MBit/s für 35€ monatlich zu bestellen. Es
gibt keine Gegenstimmen oder Enthaltungen.

Valodim und rohieb erklären sich bereit, am Tag der Inbetriebnahme des
Anschlusses anwesend zu sein.

\subsection{Mitgliedsanträge}
\vote{Mitgliedsanträge}{5}{0}{0}
Es wird ohne Gegenstimmen und Enthaltungen beschlossen, folgenden eingegangenen
Mitgliedsanträgen zuzustimmen:
\begin{itemize}
 \item 2011-12-20: Henning Hasemann
 \item 2011-12-20: Dominik Steckermeier
 \item 2011-12-28: Steffen Arntz
 \item 2012-01-02: René Stegmaier
 \item 2012-01-03: Jörg Ehring
 \item 2012-01-06: Daniel Sturm
 \item 2012-01-08: Anton Tranelis
 \item 2012-01-08: Roman Krüger
 \item 2012-01-08: Dominik Riebeling
 \item 2012-01-21: Arvid Grimm
 \item 2012-01-21: Matthias Uschok
 \item 2012-01-21: Mark Schnalke
 \item 2012-01-19: Philipp Specht
 \item 2012-01-23: Marek Möckel
\end{itemize}

%%%%%%%%%%%%
%% TOP  1 %%
%%%%%%%%%%%%
\section{Finanzielles}
\subsection{Erstattung 61€ für Miete eines Transporters}
\vote{Erstattung 61€ Transporter-Miete}{5}{0}{0}
Es wurde von einem Mitglied der Antrag gestellt, 50€ für einen Transporter der
Autovermietung Schröder zum Transport von Möbeln in den neu gemieteten
Hackerspace zu erstatten. Es hat sich aber gezeigt, dass der ausgegebene Betrag
mit 61€ höher als der im Antrag gestellte Betrag war. Der Vorstand beschließt
einstimmig und ohne Enthaltungen, 61€ zu erstatten.
 
\subsection{Erstattung 7{,}47€ Wandgarderobe}
\vote{Erstattung 7{,}47€ Garderobe}{5}{0}{0}
Von einem Mitglied wurde Antrag auf Erstattung von 7{,}47€ für den Kauf und das
Anbringen einer Wandgarderobe gestellt. Der Vorstand beschließt einstimmig und
ohne Enthaltungen die Erstattung des Betrags.

\subsection{Erstattung 34{,}92€ für Baumaterial}
\vote{Erstattung 34{,}92€ für Baumaterial}{5}{0}{0}
Für Baumaterial (Dübel, Winkel für die Leinwand, Schrauben, Lüsterklemmen,
Schlauch für die Spüle) wurden von einem Mitglied 34{,}92€ ausgegeben. Der
Vorstand beschließt einstimmig und ohne Enthaltungen, diesen Betrag zu
erstatten.

\subsection{Finanzieller Lagebericht}
\consent{Kontovollmacht für den Schatzmeister}
Neo Bechstein würde gerne einen finanziellen Lagebericht abliefern, er hat
jedoch bisher keinen Zugriff auf das vereinseigene Bankkonto erhalten. Valodim
und m00lean wollen am nächsten Tag zur Bank gehen und eine Vollmacht für ihn
ausstellen lassen.

Aus der Barkasse wurden bisher die Kaution von 900€ und 86€ für Getränke bei der
Einweihungsfeier bezahlt. Die Barkasse enthält somit zur Zeit etwa 300€.

\consent{Finanzieller Puffer von 4 Monatsmieten}
Valodim schlägt vor, möglichst schnell einen Puffer in Höhe der anfallenden
Mieten für den Zeitraum der Kündigungsfrist (4 Monate) anzulegen. Dieser
Vorschlag scheint den anwesenden Vorstandsmitgliedern sinnvoll. dstulle und
m00lean schlagen vor, vielleicht zu diesem Zweck ein Tagesgeldkonto zu eröffnen,
darüber herrscht aber keine Einigkeit.

\subsection{Gemeinnützigkeit}
In der Satzung wurde festgeschrieben, dass der Verein den Status der
Gemeinnützigkeit anstreben soll. Dies würde Steuerersparnisse mit sich bringen.
Valodim gibt zu bedenken, dass eine Gemeinnützigkeit vermutlich durch die
dadurch anfallenden Auflagen das Handeln einschränkt. Andere Hackerspaces in
Deutschland wären auch nicht gemeinnützig oder würden zumindest aus mehreren
Vereinen bestehen, die sich um verschiedene Aufgaben kümmern.

Es wird kein Beschluss in dieser Sache gefasst.

\subsection{Regelung der Matekasse}
\consent{Getränkekasse wird privat geführt}
Valodim erklärt sich bereit, die Getränkekasse als private Kasse zu betreuen
und sich privat um die Anschaffung und Abrechnung von Getränken zu kümmern. Es
besteht Konsens darüber.

\subsection{Mahnungen der ausstehenden Mitgliedsbeiträge}
\consent{Alle 2-3 Monate an ausstehende Mitgliedsbeiträge erinnern}
Es wird als Konsens erarbeitet, alle 2-3 Monate nach ausstehenden
Mitgliedsbeiträgen zu schauen und die entsprechenden Mitglieder dann per E-Mail
mit Kopie an die Vorstands-Mailingliste verbindlich an ihre Zahlungspflicht zu
erinnern.

Es wird auch kurz angerissen, ob es eine Fördermitgliedschaft für Firmen geben
sollte. dstulle erläutert, dass Firmen solche Mitgliedschaften anscheinend
leichter von der Steuer absetzen können. Da zu wenige Informationen vorliegen,
und die Mitgliederversammlung über eine entsprechende Satzungsänderung abstimmen
müsste, wird kein Beschluss zu diesem Thema gefasst. Valodim merkt aber an, dass
man das Thema im Hinterkopf behalten sollte.

%%%%%%%%%%%%
%% TOP  2 %%
%%%%%%%%%%%%
\section{PGP-Signierung von Vorstands-E-Mails}
Im Anschluss an die Sitzung sollen die Vorstandsmitglieder ihre PGP-Schlüssel
gegenseitig signieren, um ein Vertrauensnetz aufzubauen und sicherer per E-Mail
kommunizieren zu können.

\emph{Anmerkung: Diese gegenseitige Signatur fand nicht statt, da zu viele
Teilnehmer direkt nach der Sitzung aufbrachen.}

\emph{Es gibt eine kurze Pause von 19:05 bis 19:12.}

%%%%%%%%%%%%
%% TOP  3 %%
%%%%%%%%%%%%
\section{Homepage-/Server-Migration}
Hintergrund: Im Moment läuft die Domain \url{stratum0.org} auf m00leans Server,
dort liefert ein Webserver das MediaWiki aus. Ebenso werden dort die
Mailinglisten gehostet. Allerdings ist es schlecht, nur einen Ansprechpartner
für die Infrastruktur haben, außerdem sieht m00lean selber ein, dass er nicht
besonders viel Zeit für die Pflege des Servers hat. Valodim würde sich um den
Webserver und das Wiki kümmern.

Als Lösung bietet sich an, die Verantwortlichkeit und auch die Bezahlung eines
Servers auf den Verein zu übertragen. Es werden verschiedene Möglichkeiten
diskutiert:

\begin{enumerate}
	\item ein kleiner VServer (schnell, ab 8€)
	\item ein kleiner Dedicated Server (etwa 20€)
	\item ein großer Dedicated Server (etwa 80€, würde auch Virtualisierung
		erlauben)
	\item Diensteaufteilung auf verschiedene vorhandene private Server
\end{enumerate}

Valodim würde sich nur für einen Dedicated Server aussprechen, sobald der Verein
3 Monate lang finanziell auf stabiler Basis steht. m00lean schlägt vor, einen
kleineren VServer als Relay zu einem weiteren Server im Hackerspace selber zu
benutzen, Neo Bechsteins Fokus liegt auf Festplattenverschlüsselung, die mit
leistungsschwächeren Systemen nicht lohnenswert wäre. 

\consent{Hosting: Mail auf Neo Bechsteins Server, Web auf Valodims Server}
Als vorerst günstigste Lösung wird die dritte Option favorisiert, wobei die
Mailingliste auf Neo Bechsteins Server gehostet werden soll und das Wiki
auf Valodims Server. rohieb erklärt sich außerdem bereit, bei der Administration
des Wikis Unterstützung zu leisten.\footnote{Für eine Übersicht aller
angebotenen Netzdienste und zuständigen Personen siehe auch
\url{https://stratum0.org/wiki/Netzdienste}}

Die weitere Diskussion dreht sich um die Darstellung der Webseite. Neo Bechstein
schlägt vor, einen Blog einzurichten, um Projekte besser als im eher
anarchistisch geprägten Wiki präsentieren zu können. Es gibt aber den Einwand,
dass im Moment wahrscheinlich kein Bedarf besteht und ein toter Blog schlechter
ist als gar keiner. Zudem könnte man für Ankündigungen auch die bisherige
Hauptseite des Wikis benutzen. Auch die Einrichtung einer statischen Hauptseite,
eines Content Management Systems und eines Kalenders wird vorgeschlagen.
Generell gilt es aber vorerst, Freiwillige zu finden, die sich um die Migration
und die Pflege der Seite kümmern würden, sodass vorerst noch kein Entschluss
gefasst wird.

%%%%%%%%%%%%
%% TOP  4 %%
%%%%%%%%%%%%
\section{Strom}
Im Moment wird der Hackerspace durch den Grundversorgungstarif von BS Energy
gespeist. Davon sollte man auf jeden Fall weggehen, da es sehr viel günstigere
Tarifmodelle gibt. m00lean meint, man könne bei ominösen Anbietern etwa 60€ im
Jahr sparen, das würde sich nicht gegenüber BS Energy lohnen. Valodim findet für
den Anfang BS Energy in Ordnung, nach genauerer Feststellung des Verbrauchs kann
man dann später andere Anbieter suchen.

\vote{Strombezug von RWE}{5}{0}{0}
Nach kurzer Preisrecherche im Internet fällt RWE als günstigster Anbieter
heraus (78€ Einsparung im Jahr gegenüber BS Energy). Valodim erklärt sich
bereit, sich um das weitere Vorgehen zu kümmern. Es gibt keine Gegenstimmen und
keine Enthaltungen gegen den Tarif von RWE.

%%%%%%%%%%%%
%% TOP  5 %%
%%%%%%%%%%%%
\section{Physische Schlüssel zur Tür}
\vote{Erstattung 26{,}70€ Ersatzschlüssel}{5}{0}{0}
Neo Bechstein hat für die Anfertigung von 5 Ersatzschlüsseln für die obere Tür
wie auf der Mitgliederversammlung beschlossen 26{,}70€ ausgegeben. Die
anwesenden Vorstandsmitglieder sprechen sich einstimmig dafür aus, ihm diesen
Betrag zu erstatten.

Die einzelnen Schlüsselexemplare wurden bisher an m00lean, ktrask bzw. Pecca,
Neo Bechstein und Valodim verteilt, Valodim besitzt zusätzlich den einzigen
Schlüssel für die untere Haustür. Die restlichen zwei Schlüssel sollen zur
freien Vergabe an Mitglieder dienen.

\emph{Es gibt eine kurze Pause von 20:25 bis 20:35.}

\vote{Pfand von 30€}{2}{3}{0}
\vote{Pfand von 5€}{5}{0}{0}
Es wird über ein die Höhe eines Pfandes für Schlüssel abgestimmt. Für ein Pfand
in Höhe von 30€ sprechen sich nur 2 der anwesenden Vorstandsmitglieder aus, es
gibt 3 Gegenstimmen. Ein weiterer Vorschlag von 5€ findet volle Zustimmung. Das
Pfand für die Schlüssel ist von den bisherigen Besitzern beim Schatzmeister zu
hinterlegen. Außerdem sollen die aktuellen Besitzer der einzelnen Schlüssel
im Wiki dokumentiert werden.\footnote{siehe
\url{https://stratum0.org/wiki/Schl\%C3\%BCssel}}

\consent{Schlüsselinhaber im Wiki dokumentieren}
Außerdem wird erwartet, dass es in absehbarer Zukunft eine elektronische Lösung
für die Öffnung der Tür gibt. Falls dies der Fall ist, sollen es allen
Mitgliedern ermöglicht werden, einen elektronischen Schlüssel zu bekommen. Der
initiale Schlüssel mit dem Hausschlüssel wird in diesem Fall bei Valodim
hinterlegt bleiben.

rohieb erklärt sich außerdem bereit, Schlüsselquittungen zu drucken.\footnote{%
siehe \url{https://stratum0.org/wiki/Datei:Schl\%C3\%BCsselquittung-1x2.pdf}}

%%%%%%%%%%%%
%% TOP  6 %%
%%%%%%%%%%%%
\section{Hausordnung}
\consent{Bestimmte Grundregeln für den Hackerspace}
Es sollen bestimmte Grundregeln für den Hackerspace gefunden werden, die für die
Anwesenden gültig sind. Folgende Punkte finden allgemeine Zustimmung unter den
Anwesenden:
\begin{itemize}
  \item Um den Zweck eines Hackerspaces verwirklichen zu können, soll
		schlafenden Entitäten keine Sonderrechte eingeräumt werden.
	\item Kühlschrankinhalte sollen nach einer bestimmten Lagerzeit für die
		Allgemeinheit freigegeben werden. Dies soll Verderben und Überbefüllung des
		Kühlschranks verhindern. Als Konsens ergeben sich drei Tage nach Anfang der
		Lagerung im Kühlschrank, alle Inhalte sollen mit Datum des Lageranfangs
		versehen werden.
	\item Herumstehende Getränke, die keinem Besitzer zuzuordnen sind, sollen als
		Allgemeingut angesehen werden.
	\item Auf der Toilette soll im Sitzen uriniert werden und ein entsprechender
		Hinweis angebracht werden. Neo Bechstein schlägt vor, zusätzlich eine
		UV-Lampe anzubringen.
	\item An der Tür soll eine Liste mit zu erledigenden Dingen hängen, die beim
		Verlassen erledigt werden müssen (Fenster schließen, Heizung herunterdrehen,
		Müll mitnehmen, eigenes Geschirr spülen, etc.).
	\item Es soll im gesamten Hackerspace Rauchverbot herrschen, wie auf der
		letzten Mitgliederversammlung beschlossen.
	\item Außerdem soll eine öffentliche Einkaufsliste eingerichtet werden, wo
		für Dinge, die in einem Hackerspace immer verfügbar sein sollten, Bedarf
		angemeldet werden kann (Toilettenpapier, Schreibwaren, Geschirrspülmittel,
		etc.). Falls Entitäten Dinge von dieser Liste einkaufen sollten, kann beim
		Schatzmeister Erstattung angefordert werden.
\end{itemize}

%%%%%%%%%%%%
%% TOP  7 %%
%%%%%%%%%%%%
\section{Liste größerer Anschaffungen}
Folgende Anschaffung von Großgeräten wird allgemein als sinnvoll erachtet:
\begin{itemize}
	\item Geschirrspüler
	\item besserer Beamer
	\item Mikrofon, um z.~B. Vorträge aufnehmen zu können
\end{itemize}

Es wird kein weiterer Beschluss zur Höhe der Ausgaben oder ähnlichem gefasst,
dies soll bei Bedarf und nach Einholen eines entsprechenden Angebotes auf einer
der nächsten Sitzungen geschehen.

Außerdem wird zu Protokoll gegeben, dass das Deutsche Zentrum für Luft- und
Raumfahrt uns Möbel spenden würde und ein Besichtigungstermin am folgenden
Freitag stattfinden soll. Valodim und rohieb erklären sich bereit, die
Besichtigung vorzunehmen.

%%%%%%%%%%%%
%% TOP  8 %%
%%%%%%%%%%%%
\section{Umstellen der Sofas}
Die Sofas im Raum werden umgestellt, um Laufwege zu vereinfachen. Es wird
allerdings keine zufriedenstellende Lösung gefunden, die einem Durchgangsraum
gerecht wird. Es wird aus Faulheit darauf verzeichtet, den Ursprungszustand
wieder herzustellen, Valodim weist darauf hin, dass eine ausführende Entität
gegenüber einer nicht ausführendenden Entität immer im Recht ist.

\emph{m00lean verlässt die Sitzung gegen 21:25.}

%%%%%%%%%%%%
%% TOP  9 %%
%%%%%%%%%%%%
\section{Kommunikation mit Makler/Vermieter}
Bezüglich des Wasserschadens im Laborraum hatte Valodim mit dem Vermieter
telefoniert, eine Mitarbeiterin war vorbeigekommen, hatte den Schaden zur
Kenntnis genommen und versprach, sich zu melden. Bisher ist nichts dergleichen
geschehen.

Zum Briefkasten fehlt uns immer noch der Schlüssel, es scheint auch keiner der
uns ausgehändigten Schlüssel zu passen.

Bisher haben wir vom Vermieter noch keine Kontodaten erhalten, wo die Miete hin
überwiesen werden soll.

Valodim erklärt sich bereit, ein zweites Mal wegen dieser Punkte beim Vermieter
anzurufen und notfalls mit Neo Bechstein persönlich bei dessen Büro
vorbeizugehen. Neo Bechstein schlägt auch vor, Mietminderung wegen des
Wasserschadens zu fordern.

%%%%%%%%%%%%
%% TOP 10 %%
%%%%%%%%%%%%
\section{Werbeschild an der Haustür}
\consent{Cooles Werbeschild anbringen}
An der Hauswand neben der Haustür hängt immer noch das alte Werbeschild vom
Vormieter. Die anwesenden Vorstandsmitglieder beschließen einstimmig, dass sie
ein cooles Werbeschild wollen. Es müssen weitere Informationen über Angebote,
nötige Abmessungen etc. eingeholt werden.

\begin{description}
	\item[Sitzung geschlossen] um 21:45
\end{description}
\end{document}
