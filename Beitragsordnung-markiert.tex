\documentclass[a4paper,12pt]{scrartcl}
\usepackage[utf8]{inputenc}
\usepackage[T1]{fontenc}
\usepackage[ngerman]{babel}
\usepackage{textcomp}  % für Euro-Symbol
\usepackage[legal]{labdoc}
\usepackage{xcolor}

\title{Beitragsordnung des Freies~Labor~e.~V.}
\date{Entwurf \today}

\begin{document}
\maketitle

\section{Beitragssätze}
\begin{enumerate}
  \item Der reguläre Mitgliedsbeitrag für ordentliche Mitglieder beträgt 20€
    pro Monat. Fördermitglieder zahlen einen frei wählbaren Beitrag von
    mindestens 30€ pro Jahr.
  \item Schülerinnen und Schüler, Studierende, Auszubildende, 
    Empfängerinnen/Empfänger von Sozialgeld oder Arbeitslosengeld~II 
    einschließlich Leistungen nach §~22 ohne Zuschläge oder nach §~24 des
    Zweiten Buchs des Sozialgesetzbuchs (SGB~II), sowie Empfängerinnen und 
    Empfänger von Ausbildungsförderung nach dem
    Bundesausbildungsförderungsgesetz (BAföG) haben die Möglichkeit, für die
    ordentliche Mitgliedschaft einen ermäßigten Beitrag von 
    \colorbox{yellow}{10€} pro Monat zu zahlen. Ein entsprechender Nachweis muss
    dem Vorstand auf Verlangen zugänglich gemacht werden.
  \item Sollte ein ordentliches Mitglied aus finanziellen Gründen den
    Mitgliedsbeitrag nicht aufbringen können, kann dieses beim Vorstand einen
    Antrag auf Ermäßigung oder Befreiung stellen. Diese gilt für maximal ein
    Jahr und kann dann durch einen neuen Antrag erneuert werden.
  \item Alle Mitglieder werden ermutigt, im Rahmen ihrer Möglichkeiten eine
    regelmäßige Spende an den Verein zu entrichten. Empfohlen wird eine Spende
    in Höhe von 1\% des Bruttoeinkommens.
\end{enumerate}

\section{Fälligkeit}
\begin{enumerate}
  \item Der Mitgliedsbeitrag wird jeweils zum ersten Werktag eines jeden Monats
    im Voraus bzw. mit der Annahme des Aufnahmeantrags für den laufenden Monat
    fällig.
  \item Auf Wunsch des Mitglieds ist auch vierteljährliche, halbjährliche oder
    jährliche Zahlungsweise zum ersten Werktag des jeweiligen Zeitraums im
    Voraus möglich.
\end{enumerate}

\section{Zahlungsweise}
\begin{enumerate}
  \item\label{item:ueberweisung} Die Zahlung des Mitgliedsbeitrages kann per
    Überweisung (z.~B. Dauerauftrag) oder per SEPA-Lastschrifteinzug erfolgen.
    Für den Einzug per SEPA-Lastschrift muss dem Vorstand ein
    SEPA-Lastschriftmandat in Schriftform vorliegen. Eventuell anfallende
    Gebühren durch Rücklastschrift, die ein Mitglied selbst zu verschulden hat,
    werden dem Mitglied in Rechnung gestellt.
  \item Alternativ zu Abs.~\ref{item:ueberweisung} kann auch eine Barzahlung an
    den Schatzmeister/die Schatzmeisterin erfolgen, sofern diese/dieser zum 
    entsprechenden Zeitpunkt dazu bereit ist.
\end{enumerate}

\section{Aufnahmegebühren}
\begin{enumerate}
  \item Aufnahmegebühren werden nicht erhoben.
\end{enumerate}

\section{Erstattungen}
\begin{enumerate}
  \item \colorbox{yellow}{Im Voraus gezahlte Mitgliedsbeiträge für noch nicht 
    laufende Monate werden}\\\colorbox{yellow}{dem Mitglied bei Beendigung der
    Mitgliedschaft erstattet. Auf Wunsch kann}\\\colorbox{yellow}{dieser Betrag 
    an den Verein gespendet werden.}
\end{enumerate}
\end{document}
% vim: set tw=80 et sw=2 ts=2:
