\documentclass[a4paper,12pt]{scrartcl}
\usepackage[utf8]{inputenc}
\usepackage[T1]{fontenc}
\usepackage[ngerman]{babel}
\usepackage{libertine} % kann man notfalls auch ignorieren, wenns nicht da ist
\usepackage{textcomp}  % für Euro-Symbol
\usepackage[legal]{stratum0doc}

\title{Beitragsordnung des Stratum~0~e.~V.}
\date{15.~Dezember~2012}

\begin{document}
\maketitle

\section{Beitragssätze}
\begin{enumerate}
  \item Der reguläre Mitgliedsbeitrag für ordentliche Mitglieder beträgt 20€
    pro Monat. Fördermitglieder zahlen einen frei wählbaren Beitrag von
    mindestens 30€ pro Jahr.
  \item Schüler, Studenten, Auszubildende, Empfänger von Sozialgeld oder
    Arbeitslosengeld~II
    einschließlich Leistungen nach §~22 ohne Zuschläge oder nach §~24 des
    Zweiten Buchs des Sozialgesetzbuchs (SGB~II), sowie Empfänger von
    Ausbildungsförderung nach dem Bundesausbildungsförderungsgesetz (BAföG)
    haben die Möglichkeit, für die ordentliche Mitgliedschaft einen ermäßigten
    Beitrag von 12€ pro Monat zu
    zahlen. Ein entsprechender Nachweis muss dem Vorstand auf Verlangen
    zugänglich gemacht werden.
  \item Sollte ein ordentliches Mitglied aus finanziellen Gründen den
    Mitgliedsbeitrag nicht
    aufbringen können, kann dieses beim Vorstand einen Antrag auf Ermäßigung
    oder Befreiung stellen. Diese gilt für maximal ein Jahr und kann dann durch einen
    neuen Antrag erneuert werden.
  \item Alle Mitglieder werden ermutigt, im Rahmen ihrer Möglichkeiten eine
    regelmäßige Spende an den Verein zu entrichten. Empfohlen wird eine Spende
    in Höhe von 1\% des Bruttoeinkommens.
\end{enumerate}

\section{Fälligkeit}
\begin{enumerate}
  \item Der Mitgliedsbeitrag wird jeweils zum ersten Werktag eines jeden Monats
    im Voraus bzw. mit der Annahme des Aufnahmeantrags für den laufenden Monat
    fällig.
  \item Auf Wunsch des Mitglieds ist auch vierteljährliche, halbjährliche oder
    jährliche Zahlungsweise zum ersten Werktag des jeweiligen Zeitraums im
    Voraus möglich.
\end{enumerate}

\section{Zahlungsweise}
\begin{enumerate}
  \item\label{item:ueberweisung} Die Zahlung des Mitgliedsbeitrages kann per
    Überweisung (z.~B. Dauerauftrag) oder per SEPA-Lastschrifteinzug erfolgen. Für
    den Einzug per SEPA-Lastschrift muss dem Vorstand ein SEPA-Lastschriftmandat in
    Schriftform vorliegen. Eventuell anfallende Gebühren durch Rücklastschrift,
    die ein Mitglied selbst zu verschulden hat, werden dem Mitglied in Rechnung
    gestellt.
  \item Alternativ zu Abs.~\ref{item:ueberweisung} kann auch eine Barzahlung an den
    Schatzmeister erfolgen, sofern dieser zum entsprechenden Zeitpunkt dazu
    bereit ist.
\end{enumerate}

\section{Aufnahmegebühren}
\begin{enumerate}
  \item Aufnahmegebühren werden nicht erhoben.
\end{enumerate}

\section{Erstattungen}
\begin{enumerate}
  \item Im Voraus gezahlte Mitgliedsbeiträge für noch nicht laufende Monate
    werden dem Mitglied auf Wunsch bei Beendigung der Mitgliedschaft erstattet.
    Antrag zur Erstattung muss innerhalb von 6~Wochen nach Beendigung der
    Mitgliedschaft erfolgen.
\end{enumerate}
\end{document}
% vim: set tw=80 et sw=2 ts=2:
