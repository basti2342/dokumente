\documentclass[a4paper,12pt]{scrartcl}
\usepackage[utf8]{inputenc}
\usepackage[T1]{fontenc}
\usepackage[ngerman]{babel}
\usepackage{libertine} % kann man notfalls auch ignorieren, wenns nicht da ist
\usepackage{textcomp} % für €

\renewcommand{\labelenumi}{(\arabic{enumi})}
\renewcommand{\labelitemi}{--}
\makeatletter\renewcommand*\thesection{TOP\ \@arabic\c@section:}\makeatother
\setcounter{section}{-1}

\title{Vorstandssitzung des Stratum~0~e.~V.}
\date{30. August 2011}

\begin{document}
\maketitle

\begin{description}
 \item[Anwesend:] Valodim (Vorsitzender) \\
    m00lean (stv.~Vorsitzender) \\
    n30\_83c45731n (Schatzmeister) \\
    rohieb (Beisitzer) \\
    OrtwinR1 (Beisitzer) \\
    larsan (Beisitzer) \\
    DooMMasteR (Gast ohne Stimmrecht) \\
    Pecca (Gast ohne Stimmrecht)
  \item[Veranstaltung eröffnet] durch Valodim um 19:20
  \item[Protokollführer:] Roland Hieber
\end{description}

\section{Bericht des Vorsitzenden zum Fortschritt der Eintragung}
Der Notar hat Valodim eine beglaubigte Kopie seines Briefes an das Amtsgericht
bzgl. unserer Registereintragung zukommen lassen. Valodim lässt uns diese Kopie
einsehen. Weiterhin stellt der Notar für seine Dienste ein Entgelt von 26,78€ in
Rechnung. Der Vorstand spricht sich mit einer Gegenstimme dafür aus, den
Schatzmeister mit der Rechnungsbegleichung zu beauftragen.

Das Amtsgericht hat dem Vorsitzenden außerdem mitgeteilt, dass ein
Kostenvorschuss von 53€ für die Registereintragung nötig ist. Eine Befreiung
von diesem Betrag kommt nur in Betracht, wenn ein Nachweis der
Gemeinnützigkeit durch das Finanzamt gegeben ist. Der Betrag muss innerhalb von
einem Monat ab letzter Woche beglichen werden. Da die Gemeinnützigkeit beim
Finanzamt beantragt werden muss und nicht rückwirkend anerkannt werden kann,
könnte es unter Umständen geschehen, dass uns dieser Betrag vom Amtsgericht
nicht erstattet wird. Eine Abstimmung führt einstimmig zu dem Entschluss, dass
die Zahlung von 53€ genehmigt wird.

\section{Mitgliedsanträge}
Folgende Mitgliedsanträge wurden mit Wirkung zum 30.~08.~2011 einstimmig
angenommen:
\begin{itemize}
 \item Stefan Voss
 \item Stephan Christann
 \item Ann-Kathrin Christann
 \item Mathias Erdmann (ermäßigt durch Nachweis)
 \item Heinrich Schmidt
 \item Rebecca Husemann
 \item Jannik Winkel
\end{itemize}

Der Antrag von \emph{bfett} wurde einstimmig abgelehnt, da kein Realname vorlag.

\section{Alte Mitglieder}
n30\_83c45731n merkt an, dass wir ein vernünftiges Programm zur Mitgliederverwaltung und
Konto brauchen, das sich auch an Vorgaben zwecks Steuer usw. hält. Auf
Windows sind ihm eine Menge an Programmen bekannt, auf Linux keines, das gut
genug scheint. Valodim wirft ein, jemand sollte sich drum kümmern, dass es
Liste mit Vorschlägen von Software (notfalls etwas selbst geschriebenes) gibt,
und schlägt vor, über die Mailingliste Vorschläge zu sammeln. n30\_83c45731n erklärt
sich bereit, sich darum zu kümmern.

Die Ermittlung der aktuellen Mitglieder sowie der aus Stratum Epsilon und die
Entscheidung, welche Daten wir von unseren Mitgliedern haben müssen, wird
einstimmig auf einen späteren Zeitpunkt vertagt, da die entsprechenden
Mitgliederlisten noch beim Amtsgericht liegen. n30\_83c45731n übernimmt die Aufgabe,
sich alsbald darum zu kümmern.

\section{Konto}
Es wurde über eine geeignete Bank für ein Vereinskonto beraten. Gegen die
Ethikbank stehen die Argumente, dass sie warscheinlich einen "`Hackerverein"'
nicht unterstützen würde, und dass sie eine ziemlich kleine Bank zu sein
scheint. m00lean spricht sich für die Braunschweigische Landessparkasse aus, da
sie lokal ansässig ist und somit persönliche Beratung ermöglicht, außerdem
scheint sie lokale Vereine durch das günstige Angebot von Tagungsräumen zu
unterstützen.

Auf jeden Fall ist der Konsens, dass die Bank der Wahl möglichst ein kostenloses
Girokonto für Vereine anbieten sollte. In dem Punkt rät n30\_83c45731n von der
Kommerzbank ab. Weiterhin kann die Kontoeröffnung erst nach der Eintragung im
Vereinsregister und der Anerkennung der Gemeinnützigkeit geschehen.

m00lean und n30\_83c45731n erklären sich bereit, Angebote von Banken einzuholen. Der
Vorstand spricht sich einstimmig dafür aus.

\section{Sponsoren}
Drahflow hatte vor einiger Zeit angeregt, pizza.de als Sponsor zu werben,
es hat sich aber noch nichts in dieser Richtung ergeben. Jemand seriös
aussehendes sollte sich darum kümmern.

Es wurde überlegt, die Arbeitsgruppe Werbung mit der Suche nach Sponsoren zu
beauftragen. Andererseits sollte sich diese Arbeitsgruppe eher auf
Mitgliederwerbung konzentrieren, und einen eigenen Arbeitsgruppe für die Suche
nach Sponsoren soll gegründet werden. Valodim schlägt vor, das Thema auf die
Mailingliste zu tragen und dort nach Freiwilligen und/oder Firmenkontakten zu
suchen. Er erklärt sich bereit, eine entsprechende Mail zu schicken, und sieht
keinen Grund, das Thema bis nach der Eintragung aufzuschieben.

m00lean erwähnt einen Stammtisch der Gruppe "`IT-Region 38"', einer Verbindung
von IT-Firmen aus dem Großraum Braunschweig, und er hat in dem
Zusammenhang Kontakte, die er nutzen könnte. Auf erwähntem Stammtisch könnte man
sich mit ein paar Leuten, die nach Programmieren aussehen, einmischen. Valodim
merkt hierzu an, dass rohieb ohne Bart wie ein Windows-Benutzer aussieht.\\
\\
Es gibt eine kurze Pause von 10 Minuten bis 20:30.

\section{Space}
m00lean hat sich um die mögliche Location im Keller an der Beckenwerkerstraße 2
bemüht. Eine Versorgung mit Frisch- und Abwasser wäre im Rahmen von etwa 1000€
möglich, wovon der Vermieter 500€ beisteuern würde. Es ist in seiner Meinung
aber nötig, ein (mündliches) Gutachten von einem unabhängigen Experten wegen der
feuchten Wände und dem wassergefüllten Loch im Boden einzuholen. Valodim meint,
dass man unter Umständen professionelle Luftentfeuchter mieten könnte, um dem
Problem Herr zu werden. m00lean hat mit dem Vermieter kontakt und erklärt sich
bereit, sich kurzfristig um die Entfeuchtung zu kümmern und einen
Besichtigungstermin im kleineren Rahmen (2-3 Leute) mit dem Vermieter zu
vereinbaren. Ein Budget von 150€ für ein Gutachten wurde einstimmig beschlossen.

Des Weiteren gibt es laut m00lean die Möglichkeit, dort (V)DSL legen zu lassen.
Außerdem wären Wasseranschlüsse für eine Küche gut zu haben, sowie eine Lüftung
für die Toilette. Er regt dies beim Treffen mit dem Vermieter an.

Der Begriff "`Stratumsphäre"' wurde in den Raum geworfen und bietet sich als
Name für die Location an. Valodim regt die Beschaffung von Portalen an, um
notfalls den Lastenaufzug umgehen zu können.

\section{Workshops}
Der Großteil des Vorstandes ist dagegen, Workshops abzuhalten, bevor eine
Räumlichkeit existiert. Die Arbeit soll lieber erstmal in die schnelle
Bereitstellung einer Räumlichkeit investiert werden. Falls Mitglieder
allerdings selbst Workshops oder Vorträge ohne ein Vereinsheim organisieren
wollen, steht dem nichts entgegen.

\section{Werbung}
Aus dem gleichen Grund ist Werbung zum jetzigen Zeitpunkt nachrangig. Auch hier
gilt, dass die Mitglieder gerne in eigene Werbung investieren können,
allerdings gibt es zum jetzigen Zeitpunkt keine Kostenerstattung durch den
Verein. Eine entsprechende Arbeitsgruppe um blinry wurde auch schon
gegründet, es gibt aber noch keine Ergebnisse. Insbesondere muss gewartet
werden, bis ein Logo gefunden ist.

m00lean hatte auf der Mailingliste einen Logowettbewerb ausgeschrieben, der am
31. August endet. Danach soll eine offene Abstimmung mit Namensnennung unter den
Mitgliedern ein Meinungsbild ergeben, wobei Namen, die dem Vorstand nicht
bekannt sind, nach 7 Tagen über die Mailingliste aufgefordert werden, sich zu
identifizieren. Notfalls werden solche Stimmen als ungültig anerkannt. Der
Vorstand beschließt dann aufgrund des Meinungsbildes ein Vereinslogo. Valodim
erklärt sich bereit, eine Webseite für die Abstimmung aufzusetzen.

\section{Sonstiges}
Die nächste Mitgliederversammlung soll sich mit der Miete des Raumes
beschäftigen, und mindestens nach der erfolgten Eintragung stattfinden.

Um die aktuellen Mitglieder bei der Stange zu halten, schreibt Valodim über die
Mailingliste eine Statusmail zum aktuellen Stand.

m00lean regt einen privaten IRC-Channel ohne Logging an. Die
Vorstandsmitglieder sind darüber geteilter Meinung, und es wurde kein
Entschluss gefasst.

\begin{description}
 \item[Veranstaltung geschlossen] durch Valodim um 21:27.
\end{description}

\end{document}
