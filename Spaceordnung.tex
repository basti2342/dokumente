\documentclass[12pt,a4paper]{scrartcl}

\usepackage[ngerman]{babel}
\usepackage[T1]{fontenc}
\usepackage[utf8]{inputenc}
\usepackage[legal]{stratum0doc}
\usepackage{libertine}    % nicht unbedingt notwendig
\usepackage{textcomp}

\title{The Hackerspace Agreement}
\date{\today}

\begin{document}
\maketitle

\section*{Präambel}
{\itshape
Wir setzen uns zum Ziel, dass jeder Hacker, jedes Mitglied des Stratum~0~e.~V.,
jeder Besucher und jede anderweitig nicht näher spezifizierte Lebensform
(zusammenfassend nachfolgend als "`Entitäten"' bezeichnet) sich in unserem
Hackerspace, der \emph{Stratumsphäre}, (nachfolgend als "`Space"' bezeichnet)
wohlfühlen soll. Um die zwischenentitätlichen Konflikte im Vereinsleben
möglichst gering zu halten, sollen die folgenden Punkte als Kodex
für den Umgang der Entitäten miteinander und für die Anwesenheit im Space
dienen. Mit der Nutzung des Spaces akzeptieren die benutzenden Entitäten diesen
Kodex.
}

\section{Allgemeines}
\label{sec:allgemeines}
\begin{enumerate}
  \item Alle Einrichtungsgegenstände, elektronischen Geräte und vor Ort
    befindlichen Entitäten im Space sind pfleglich zu behandeln.

  \item Im gesamten Space gilt ein Rauchverbot für alle rauchbaren und nicht
    rauchbaren Substanzen. Ausgenommen davon sind elektronische Bauteile, die
    aus eigenem Antrieb zu rauchen anfangen.

  \item Aufräumende oder putzende Entitäten werden als Putz- oder Aufräumentität
    bezeichnet. Putz- oder Aufräumentitäten genießen Heldenstatus und haben im
    Rahmen ihrer Putz- oder Aufräumtätigkeit Weisungsrecht gegenüber
    nichtaufräumenden und nichtputzenden Entitäten.

  \item\textbf{Lebensmittel (einschließlich Getränke)}
    \label{item:lebensmittel}\begin{enumerate}
    \item Offensichtlich besitzerlose Lebensmittel sind Allgemeingut und dürfen
      ohne Rückfrage entsorgt werden. Um dies zu verhindern, kann die
      Lebensmittelverpackung vor Ort vom Besitzer gekennzeichnet werden, sodass
      der Besitzer von anderen Entitäten leicht erkennbar ist. Hierzu sei
      angemerkt, dass insbesondere die Bezeichnungen "`Club-Mate"', "`Cola"',
      "`Milka"' und andere Markennamen (in allen möglichen Abwandlungen) und die
      ursprüngliche Beschriftung der Lebensmittelverpackung nicht als Besitz
      implizierende Kennzeichnung angesehen werden.
    \item Lebensmittel, die offensichtlich längere Zeit nicht angefasst wurden,
      sind als besitzerlos anzusehen.
    \item Entitäten, die besitzerlose Lebensmittel oder leere
      Lebensmittelverpackungen entsorgen, sind als Putz- oder Aufräumentität
      anzusehen.
  \end{enumerate}

  \item\textbf{Einkäufe}\begin{enumerate}
    \item Falls Kleinigkeiten (Toilettenpapier, Büromaterial, Müllbeutel, etc.)
      fehlen, werden sie auf die entsprechende Liste im Flur geschrieben.
    \item Falls sich eine nichtleere Menge Kleinigkeiten auf der entsprechenden
      Liste im Flur befindet, dürfen diese von jeder Entität angeschafft werden,
      die sich dafür bereit erklärt.
    \item Entitäten, die sich bereit erklären, Kleinigkeiten für Vereinszwecke
      anzuschaffen, werden als Einkaufsentitäten bezeichnet. Einkaufsentitäten
      genießen Heldenstatus.
    \item Die Erstattung von Beträgen, die von Einkaufsentitäten für
      Vereinszwecke ausgelegt wurden, erfolgt beim Schatzmeister gegen Vorlage
      eines Kassenbons, auf dem sich nur die für Vereinszwecke angeschafften
      Kleinigkeiten befinden dürfen.
    \item Die selbstständige Erstattung von Beträgen aus der Mate-Kasse ist
      ausdrücklich verboten!
  \end{enumerate}

  \item\textbf{Hardwareobjekte im Space}\begin{enumerate}
    \item Mitgliedern steht es frei, im Rahmen ihrer Tätigkeit an Projekten im
      Space Hardwareobjekte jeglicher Form dort aufzubewahren, sofern Platz zur
      Aufbewahrung vorhanden ist.
    \item Im Space aufbewahrte Hardwareobjekte sollten mit Edding,
      Post-It, Dymo, o.\,ä. gekennzeichnet werden, sodass der Besitzer für alle
      Mitglieder erkennbar ist.
    \item Der Space soll jedoch keine Schrotthalde werden. Falls im Space
      aufbewahrte Hardwareobjekte allgemein als unerwünscht angesehen werden,
      kann der Besitzer vom Vorstand aufgefordert werden, diese Objekte aus dem
      Space zu entfernen.
    \item Geschieht die Entfernung nicht innerhalb eines Monats nach der
      Aufforderung durch den Vorstand, kann der Vorstand das Hardwareobjekt in
      das Eigentum einer anderen Entität übergeben.
    \item Entitäten, die unerwünschte Hardwareobjekte entfernen, werden als
      Aufräumentitäten angesehen.
  \end{enumerate}

  \item\textbf{Verlassen des Spaces}\begin{enumerate}
    \item Jede Entität, die den Space verlässt, wäscht vorher mindestens das von
      ihr benutzte Geschirr ab und bringt falls nötig auf dem Weg nach unten den
      Müll in die dafür vorgesehenen Müllcontainer auf dem Parkplatz. Falls
      Geschirr abgewaschen oder Müll entsorgt wird, ist die diese Aktion
      ausübende Entität als Putz- oder Aufräumentität anzusehen.
    \item Die letzte Entität, die den Space verlässt, dreht die Heizung
      herunter, schließt alle Fenster, schaltet alle Lampen und elektronischen
      Geräte aus, die nicht dauerhaft laufen müssen, und zieht die Tür hinter
      sich zu.
  \end{enumerate}
\end{enumerate}

\section{Küche}
\begin{enumerate}
  \item Die selbstständige Aneignung von Getränken durch anwesende Entitäten ist
    nach Zahlung des entsprechenden Preises in die Mate-Kasse gestattet.
  \item Nach Benutzung von Küchengeräten sind diese zu reinigen. Dazu gehört,
    nach Benutzung der Kaffeemaschine den gebrauchten Kaffeefilter zu entsorgen.
  \item\textbf{Kühlschrank}\begin{enumerate}
    \item Alle Kühlschrankinhalte sind grundsätzlich Allgemeingut und dürfen
      ohne Rückfrage vernichtet werden.
    \item An Kühlschrankinhalten kann eine Kennzeichnung (Post-It, Edding
      o.\,ä.) mit Name des Besitzers und Anfangsdatum der Lagerung
      angebracht werden. Dementsprechend gekennzeichnete Inhalte sind als
      Eigentum des Besitzers zu betrachten, von einer Vernichtung sollte in
      diesem Fall Abstand genommen werden. Falls das Datum mehr als drei Tage
      zurückliegt, entfällt dieser Sonderstatus wieder.
    \item Für Nicht-Kühlschrankinhalte ist entsprechend
      \ref{sec:allgemeines}~Abs.~\ref{item:lebensmittel} anzuwenden.
  \end{enumerate}
\end{enumerate}

\section{Bad}
\begin{enumerate}
  \item Das Bad ist in einem sauberen Zustand zu halten. Dazu gehört, selbst
    verursachte Unsauberkeiten zu entfernen. Entitäten, die Unsauberkeiten im
    Bad entfernen, werden als Putzentitäten angesehen.

  \item Sitzpinkeln is mandatory. Bei Nichtbeachtung dieser Regelung wird
    Pinkelverbot ausgesprochen.
\end{enumerate}

\section{Überlassen des Spaces an Mitglieder}
\begin{enumerate}
  \item Der Space kann nach Absprache und Verfügbarkeit von Mitgliedern
    für private Veranstaltungen (Geburtstagsfeiern etc.) genutzt werden.
  \item Für private Veranstaltungen ist vor der Veranstaltung beim Vorstand ein
    Pfand von 50€ zu hinterlegen.
  \item Das Pfand wird an das entsprechende Mitglied zurückgezahlt, wenn es die
    Auswirkungen der privaten Veranstaltung auf die Ordnung und Sauberkeit des
    Spaces bis 20:00 am Tag nach Beginn der Veranstaltung beseitigt hat.
    War dies nicht der Fall, wird das Pfand an diejenige Entität ausgezahlt, die
    die entstandenen Auswirkungen beseitigt hat.
\end{enumerate}

\section{Schlussbestimmungen}
\begin{enumerate}
  \item Das Hausrecht wird ausgeübt durch den Vorstandsvorsitzenden, den
    stellvertretenden Vorstandsvorsitzenden, den Schatzmeister, einen Beisitzer
    des Vorstandes, oder ein Mitglied des Vereins, in dieser Rangfolge und je
    nach Anwesenheit der jeweiligen Personen.

  \item Um den Zweck eines Hackerspaces erfüllen zu können, haben schlafende
    Personen keinen Anspruch auf Ruhe oder Sonderbehandlung.

  \item Die oben genannten Bestimmungen dieser Ordnung sind als unveränderlich
    anzusehen.

  \item Diese Ordnung tritt mit ihrer Verkündung durch den Vorstand in Kraft.
\end{enumerate}

\end{document}
% vim: set tw=80 et sw=2 ts=2:
