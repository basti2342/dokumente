\documentclass[a4paper,12pt]{scrartcl}
\usepackage[utf8]{inputenc}
\usepackage[T1]{fontenc}
\usepackage{ae,aecompl}
\usepackage[ngerman]{babel}
\usepackage{libertine} % kann man notfalls auch ignorieren, wenns nicht da ist
\setlength{\parindent}{0pt}

\renewcommand{\labelenumi}{(\arabic{enumi})}
\renewcommand{\labelitemi}{--}
% \makeatletter\renewcommand*\thesection{\@arabic\c@section}\makeatother

\title{Gr\"undungsprotokoll Stratum~0~e.\,V.}
\date{}

\begin{document}
\maketitle

\quad Nach Einladung vom 01.~Juli~2011:\\
\textbf{Beginn der Gr\"undung 16:45 Uhr am 23.~Juli~2011 im Dialog, Braunschweig}

\quad\\

Wahl des Versammlungsleiters: Vincent Breitmoser (per Handzeichen), Kandidat nimmt Wahl an und
übernimmt fortan die Versammlungsleitung \\

Wahl eines Protokollanten: Roland Hieber (per Handzeichen), Kandidat nimmt Wahl an

\section{Begr\"u\ss{}ung}
\begin{itemize}
    \item Der Versammlungsleiter eröffnet die Sitzung
    \item Uhrzeit: 16:50
    \item Datum: 23.07.2011
    \item Ort: Restaurant Dialog, Rebenring 48, Braunschweig
\end{itemize}

\section{Feststellung der Anzahl der stimmberechtigten Teilnehmer}
\begin{itemize}
    \item 22 anwesende Teilnehmer haben das 18. Lebensjahr mind. abgeschlossen
\end{itemize}

\pagebreak

\section{Genehmigung einer Tagesordnung}
Vorgeschlagene Tagesordnung:

\begin{itemize}
    \item Wahl des Versammlungsleiters
    \item Wahl eines Protokollanten
    \item[1.] Begrüßung
    \item[2.] Feststellung der Anzahl der stimmberechtigten Teilnehmer
    \item[3.] Genehmigung der Tagesordnung
    \item[4.] Beratung und Verabschiedung einer Satzung
    \item[5.] Wahlen des Vorstands
    \item[6.] Wahlen der Kassenprüfer
    \item[7.] Weitere Vorgehensweise
\end{itemize}

Genehmigt durch Abstimmung per Handzeichen

\section{Beratung und Verabschiedung einer Satzung}

\begin{itemize}
    \item Wurde im Vorfeld ausgiebig diskutiert. Die endültige Fassung wird zur
        Kenntnisnahme
rumgereicht.
    \item Alle anwesenden Teilnehmer beschließen mit ihrer geleisteten
        Unterschrift unter dem Satzungsoriginal die Gründung des Vereins
        "`Stratum 0"'.
\end{itemize}

\pagebreak

\section{Wahlen des Vorstands}

Wahl eines Wahlleiters: Ortwin Regel (per Handzeichen), Kandidat nimmt Wahl an

\subsection{Wahl des Vorsitzenden}

Kandidaten: Vincent Breitmoser (Kandidat 1), Michael Klug (Kandidat 2) \\

Beginn des Wahlgangs: 18:30 \\
Ende des Wahlgangs: 18:36

\begin{itemize}
    \item Kandidat 1: Vincent 19 Stimmen
    \item Kandidat 2: Michael 16 Stimmen
\end{itemize}

Gewählt wurde Vincent Breitmoser. Der Kandidat nimmt die Wahl an.

\subsection{Wahl des stellvertretenden Vorsitzenden}

Kandidaten: Michael Klug (Kandidat 1), Roland Hieber (Kandidat 2), Ortwin Regel
(Kandidat 3) \\

Wahlleiter für diesen Wahlgang: Nico Grasshoff, einstimmig beschlossen \\

Beginn des Wahlgangs: 18:45 \\
Ende des Wahlgangs 18:53

\begin{itemize}
    \item Kandidat 1: Michael 19 Stimmen
    \item Kandidat 2: Roland 9 Stimmen
    \item Kandidat 3: Ortwin 13 Stimmen
\end{itemize}

Gewählt wurde Michael Klug. Der Kandidat nimmt die Wahl an.

\subsection{Wahl des Schatzmeisters}

Kandidaten: Julien Jassmann (1. Kandidat), Lars Andresen vorgeschlagen (lehnt
ab), Roland Hieber (2. Kandidat) \\

Beginn des Wahlgangs: 18:56 \\
Ende des Wahlgangs: 18:58

\begin{itemize}
    \item Kandidat 1: Julien Jassmann 19 Stimmen
    \item Kandidat 2: Roland 13 Stimmen
\end{itemize}

Gewählt wurde Julien Jassmann. Der Kandidat nimmt die Wahl an.

\subsection{Wahl der Beisitzer}

Gew\"ahlt werden bis zu drei Kandidaten jeweils mit den meisten Stimmen,
mindestens aber mehr als 50\%. \\

Wahlleiter f\"ur diesen Wahlgang: Juliane Schmidt, Vorschlag mit einer
Enthaltung angenommen. \\

Kandidaten: Lars Andresen (Kandidat 1), Nico Grasshoff (Kandidat 2), Roland
Hieber (Kandidat 3), Ortwin Regel (Kandidat 4), Jan-Henrik Kluth (Kandidat 5),
Björn Kalkbrenner (Kandidat 6) \\

Beginn des Wahlgangs: 19:05
Ende des Wahlgangs: 19:07

\begin{itemize}
    \item Kandidat 1: Lars Andresen 19 Stimmen
    \item Kandidat 2: Nico Grasshoff 16 Stimmen
    \item Kandidat 3: Roland Hieber 19 Stimmen
    \item Kandidat 4: Ortwin Regel 18 Stimmen
    \item Kandidat 5: Jan-Henrik Kluth 8 Stimmen
    \item Kandidat 6: Björn Kalkbrenner 8 Stimmen
\end{itemize}

Gewählt wurden Lars, Roland und Ortwin. Die Kandidaten nehmen die Wahl jeweils
an.


\section{Wahl der Kassenprüfer}

Kandidaten: Juliane Schmidt (Kandidat 1), Jan Lübbe (Kandidat 2), Jens Schicke
(Kandidat 3) \\

Beginn: 19:10 Uhr \\
Geschlossen: 19:15 Uhr

\begin{itemize}
    \item Kandidat 1: Juliane Schmidt 19 Stimmen
    \item Kandidat 2: Jan Lübbe 18 Stimmen
    \item Kandidat 3: Jens Schicke 15 Stimmen
\end{itemize}

Gewählt: Jan Lübbe und Juliane Schmidt\\

\pagebreak

\section{Weitere Vorgehensweise}
\begin{itemize}
    \item Per Handabstimmung wurde beschlossen, die Abwicklung der Eintragung
        in das Vereinsregister vom Vorstand vornehmen zu lassen.
\end{itemize}

\section{Verschiedenes}
\begin{itemize}
    \item Die Beitragsordnung wurde im Vorfeld ausreichend diskutiert. Sie
        wird rumgereicht, und per Handzeichen einstimmig angenommen.
    \item Die Wahlzettel werden dem Vorsitzenden vom Wahlleiter zur
        Archivierung übergeben.
\end{itemize}


19:37 Uhr Der Versammlungsleiter schlie\ss{}t die Versammlung.

\end{document}
% vim: set tw=80 et sw=2 ts=2:
